\documentclass[twoside,a4paper,fleqn]{report}
\usepackage[report]{Graphlet}

\begin{document}

% This file is generated automatically
\newcommand{\GraphletVersion}{1}


\title{GML: Graph Modelling Language
  \\*[3.0cm]
  {\emph{DRAFT VERSION}}
  }

\author{Michael Himsolt\thanks{        
    This research is partially supported by the Deutsche
    Forschungsgemeinschaft, Grant Br 835/6-2,
    research cluster ``Efficient Algorithms for
    Discrete Problems and Their Applications''
    }
}

\maketitle

% Abstract
% This paper defines the GML file format. GML is an efficient, platform 
% independent, expansible file format for graphs. A GML file is a 
% hierarchical list of key-value pairs. Its simple structure guarantees 
% that other file formats can effectively be converted from and to GML.


\paragraph{Acknowledgements.} I would like to thank all people on the 
file format mailing list (\url{fileformat@fmi.uni-passau.de}),
for their helpful comments in providing this paper.  Special
thanks to Martin Ramsch (\url{ramsch@fmi.uni-passau.de}) for his
ISO 8859 character set tables.

\tableofcontents

%
% Introduction
%

\chapter{Introduction}
\label{c:Introduction}

\section{File Formats}

\subsection{User vs developer perspective}

File formats do often provide tough problems both for the
software engineers who write programs and for the people who are
using them.  Software engineers want formats that store data in
an efficient manner, and are easy to read and write. Users want
way to save their data in a convenient and fast manner, and don't
want to be bothered with the choice of a specific format.

\subsection{Converters}

The consequence is that almost every graphics or desktop publishing 
system has its own file format, optimized for the needs of that 
product. This means that direct data exchange between different 
products is impossible, since the file formats will be mutually 
exclusive. So, most programs contain lots of converters that 
transform data between different formats.

\subsection{The User's Perspective}

Having lots of converters this is inconvenient for the user. First, 
it means that $O(n^{2})$ converters are ideally needed to exchange data 
between n programs. However, it is unlikely that each program can 
read and write each other format.
If programs cannot exchange data since they don't understand each 
others format, but both understand a third program, there is a still 
a way to exchange data with that format. However, this is 
inconvenient because the user has to find out which programs are 
candidates for such an operation.
Furthermore, data may be lost in a format conversion. This may be 
because the other format is simply too complicated, or even secret 
information. There is obviously no way to avoid both issues, but they 
should not be as bothering as they are.

\subsection{The Developer's Perspective}

From an engineering point of view, it is conceivable that most
programs need their own format to represent their data in an
efficient way. However, it is not so obvious that the user needs
to be bothered with this.  First, consider converters. One easy
way to get rid of them is to provide one powerful format that has
a core part which is understood by all participating
applications, and can be extended to meet a particular format's
needs.  The RTF and SGML formats for marking text are such
approaches, The PICT graphics format is a successful example for
a graphics format.  Second, consider efficiency. It is often the
case that a more generalized format is less efficient in terms of
storage space or loading time. However, one can provide the
choice of saving data either in a native, efficient format, or in
a exchange format. Many desktop publishing programs use this
approach.

\section{Intermezzo: What is a graph ?}

\begin{definition}[Graph]
A graph is a tuple
\begin{eqnarray*}
  G & =         & (V,E), \mbox{\textnormal{where}} \\
  V &           & \mbox{\textnormal{is the set of \emph{nodes}, and}} \\
  E & \subseteq & V \times V \mbox{\textnormal{is the set of edges}}
\end{eqnarray*}
We also define a mapping \emph{label} which assigns information
to nodes, edges and labels:
\begin{displaymath}
  label: G \cup V \cup E \mapsto \Sigma^{*}
\end{displaymath}
\end{definition}

\begin{notes}
  \item In graph theory, $\Sigma$ is usually a fixed alphabet;
  GML needs a more general approach and allows to attach
  arbitrary attributes to graphs, nodes and edges.
\end{notes}


%
% A Simple GML Example
%

\section{A Simple GML Example}

\begin{example}%
{e:GML:Intro:circle3}%
{A simple graph in GML (circle of three nodes)}
\begin{alltt}
graph [                     \ttcomment{Defines a new graph}
    node [                  \ttcomment{Defines a new node}
        id 1                \ttcomment{This node has the id 1}
    ]
    node [                  \ttcomment{Defines a new node}
        id 2                \ttcomment{This node has the id 2}
    ]
    node [                  \ttcomment{Defines a new node}
        id 3                \ttcomment{This node has the id 3}
    ]
    edge [                  \ttcomment{Defines a new edge}
        source 1            \ttcomment{Source is the node with the id 1}
        target 2            \ttcomment{Target is the node with the id 2}
    ]
    edge [                  \ttcomment{Defines a new edge}
        source 2            \ttcomment{Source is the node with the id 2}
        target 3            \ttcomment{Target is the node with the id 3}
    ]
    edge [                  \ttcomment{Defines a new edge}
        source 3            \ttcomment{Source is the node with the id 3}
        target 1            \ttcomment{Target is the node with the id 1}
    ]
]
\end{alltt}
\end{example}

\begin{example}%
{e:GML:ComplexExample}%
{A complex GML example}
\begin{alltt}
# This file is in version 1 of GML
Version 1

graph [

  # \ttcomment{This graph has been created by the program "demo"}
  Vendor "demo"

  # directed \ttcomment{determines whether a graph is directed (1) or not (0).}
  # \ttcomment{In a directed graph, edges are have arrows that indicate the direction.}
  directed 1

  # \ttcomment{A label is a text attached to an object}
  label "The principles of space travel"

  node [  
    id 1
    label "Earth"
    graphics [
      x 0.1
      y 0.0
      w 0.1
      h 0.1
      image "earth.gif"
    ]
  ]

  node [
    id 2    
    label "Mars"
    graphics [
      x 0.9
      y 0.0
      w 0.055
      h 0.055
      image "Mars.gif"
    ]
  ]

  edge [
    source 1        
    target 2        
  ]
]
\end{alltt}
\end{example}

Example \ref{e:GML:Intro:circle3} shows a simple graph which
consists of three nodes.  The more complex example
\ref{e:GML:ComplexExample} demonstrates how to attach text to
graphs, nodes and edges, and how to specify coordinates and
images for nodes and edges.  These example illustrates the key
elements of GML:

\begin{itemize}

  \item A GML file is made up of pairs of a key and a value.
  Examples for keys are \texttt{graph}, \texttt{node} and
  \texttt{edge}.
  
  \item The key idea behind GML is that there are some
  standard keys like graph, node and edge, and anybody is free
  to add its keys to add specific information.

  \item Values can be integers, floating point numbers, strings
  and lists, where the latter must be enclosed in square
  brackets.
  
  \item The graph in Example \ref{e:GML:ComplexExample} did not
  specify how to place the labels.  They are arranged in a
  convenient manner in Figure 2, but an application might also
  ignore graph labels and print node labels over the images.
  There is no way to prevent a program from not drawing labels,
  but we will show in sections \ref{s:GML:NodeAttributes} and
  \ref{s:GML:EdgeAttributes} how the placement of labels is specified.

\end{itemize}



%
% Design Issues
%

\section{GML Design Issues}

\subsection{Syntax: Simple or Complex ?}

A complex syntax -- like a programming language -- gives the
designer the freedom to express facts in a efficient and
easy-to-read manner. The \texttt{dot} format is a good example
for this practice. However, the price to be paid is a less simple
implementation.  A simple syntax has the advantage that the
implementation is easier, but the format is less efficient in
terms of storage space and runtime. However, there is no reason
why the format should be less powerful; everything that can be
expressed with a complex format can be expressed in a simple
format.

\textbf{Answer: Simple} We chose a simple format over a complex
one. We loose some efficiency, but we gain a much easier
implementation and thus a wider distribution.  However,
simplicity is limited: since the format needs to be universal,
some details will be slightly complex. For example, strings are
always terminated by " characters. Therefore, we need a mechanism
to deal with a quote that appears inside a string. Other issues
are maximum line length and non ASCII characters, such as German
umlauts, \"{a}.


\subsection{Data types}

Which types of values do we need to represent? The answer is: all
data types present in programming languages. This includes
numbers (both integers and floating point), boolean values,
characters, strings and composite data types such as record,
array, set and list structures.


\subsection{Constraints}

There are several external constraints which have to be
considered:

\begin{description}
  
  \item[Maximum line length] Some systems cannot handle arbitrary
  long lines without problems.  Therefore, we need to restrict
  line lengths to a size that fits all systems.
  
  \item[Character Set] Internationalization is an important issue
  these days, and the ASCII characters are no longer sufficient
  for a real application.  Therefore, we will use the ISO 8859
  character set, which is a common way to code non ASCII
  character sets within ASCII.  ISO 8859 is also used in HTML, so
  we are in good company.
  
  \item[Range of values] The range of numbers is another
  sensitive point. We will assume 32 bit signed integers and
  double precision floating point values. These should be
  supported by all current systems.
  
  This rules out other data types like unsigned integers and 64
  bit integers, but those could still be stores as strings and
  converted afterwards.  We will not assume a maximum length for
  strings, as this would be a restrictions for applications that
  store long texts in strings.

\end{description}


%
% Notes on our Notation
%

\section{Notes on our Notation}

We use a Pascal like notation with some object oriented
extensions\footnote{Graphlet's implemention of GML is in C++, but
  we feel that Pascal is better for for aesthetic reasons, and
  more people are familiar with Pascal than with the fine prints
  of C++.}. A GML file is composed of key-value pairs, which we
call objects.  An object is a parametrized data type:

\begin{alltt}
type Object(\Param{Type}) = record
    key:   Key;
    value: \Param{Type};
end;
\end{alltt}

\noindent In the following, we will use the types \texttt{Object(Integer)},
\texttt{Object(Real)}, \texttt{Object(String)} and
\texttt{Object(List(Object))}.  We will also assume that the type
of an object is available at runtime, as in the following
example:

\begin{alltt}
case type(\Param{o}) of
    Object(Integer):      {\textnormal{\emph{Action for type}}} Integer;
    Object(Real):         {\textnormal{\emph{Action for type}}} Real;
    Object(String):       {\textnormal{\emph{Action for type}}} String;
    Object(List(Object)): {\textnormal{\emph{Action for type}}} List(Object);
end
\end{alltt}


%
% Paths
%

\subsection{Paths}

The data structure Object forms a tree, where elements of type
\texttt{Object(Integer)}, \texttt{Object(Real)} and
\texttt{Object(String)} are leaves, and elements of type
\texttt{Object(List(Object)))} are inner nodes.  We will
frequently use paths in this tree to describe the location of an
object.

\begin{definition}[Path]
  Let $k_{1},k_{2},\ldots,k_{n}, n \ge 1$ be keys. The path
  \[
  .k_{1}.k_{2}.\ldots.k_{n}
  \]
  denotes all sequences of objects $o_{1},o_{2},\ldots,o_{n}$ where
  \begin{eqnarray*}
    1 \le i \le n   & : & o_{i}.\mbox{key} = k_{i} \\
    1 \le i \le n-1 & : & \mbox{type}(o_{i}.\mbox{value}) =
    \mbox{Object(List(Object))} \\
    2 \le i \le n   & : & o_{i} \in o_{i-1}.\mbox{value}
  \end{eqnarray*}
\end{definition}

\noindent Examples of paths are \texttt{.id}, \texttt{.graph.label},
\texttt{.node.label} and \texttt{.node.graphics}. A path
describes a class of sequences of objects which can start
anywhere in the object tree. We can also define paths that start
at a specific object:

\begin{definition}[Path Starting at an Object]
  Let $k_{1}.k_{2}.\ldots.k_{n}, n \ge 1$ be a path and $o$ be an
  object. Then
  \[
  \mbox{\texttt{o}}.k_{1}.k_{2}.\ldots.k_{n}
  \]
  denotes the path starting at o, that is all sequences of
  objects $o_{1},o_{2},\ldots,o_{n}$ where
  \begin{eqnarray*}
    1 \le i \le n   & : & o_{i}.\mbox{key} = k_{i} \\
    1 \le i \le n-1 & : & \mbox{type}(o_{i}.\mbox{value}) = 
    \mbox{List(Object)} \\
    2 \le i \le n   & : & o_{i} \in o_{i-1}.\mbox{value} \\
    & & o_{1} \in o.\mbox{value}
  \end{eqnarray*}
\end{definition}

\noindent Finally, we omit the leading point on a path if it
starts at the root object:

\begin{definition}[Path Starting at the Root]
  Let $k_{1},k_{2},\ldots,k_{n}, n \ge 1$ be keys.
  \[
  k_{1}.k_{2}.\ldots.k_{n}
  \]
  denotes the path
  \[
  R.k_{1}.k_{2}.\ldots.k_{n}
  \]
  where $R$ is the root object of the tree.
\end{definition}


%
% Other Graph File Formats
%

\chapter{Other Graph File Formats}

\begin{note}
  This chapter is not complete yet.
\end{note}

\section{Simple adjacency lists}

Many systems use simple adjacency lists, perhaps enriched with labels 
or coordinates. Often, an adjacency list is terminated by the end of 
the line.
While this format type is convenient and easy to use in these 
systems, it has several disadvantages for our purpose. First, it is 
not expandible. Second, labels are usually restricted to one 
character or a single word. Further, the degree of a node is limited 
on systems which do not support arbitrary line lengths.

\section{GraphEd}

GraphEd had a format which is in spirit very simpliar to the one 
which is presented in this paper. However, its syntax is more complex 
than necessary in several aspects.

\begin{verbatim}
GRAPH "" =
1 {$ NS 32 32 $} ""
 2 ""
;
2 {$ NS 32 32 $} ""
 3 ""
;
3 {$ NS 32 32 $} ""
 1 ""
;
END
\end{verbatim}

This format contains several complications

\begin{itemize}

\item There are several ways to represent lists, e.g \verb|"[ �]"|, 
\verb|"{$ � $}"| and \texttt{GRAPH}\ldots\texttt{END}.

\item Adjacency lists start with a node and end with \texttt{";"}, 
whereas the list of nodes in a graph is terminated by \texttt{"END"}.

\item Some syntax elements are superficial, like the \texttt{"="} 
after the keyword \texttt{GRAPH}.

\end{itemize}

On the other hand, the format supported generic attributes (inside 
\verb|"{$ � $}"|) which were very similar to the one we propose. The main 
difference was that GraphEd's attributes had a key and a list of 
values, where we will only have one value per key. Of course 
GraphEd's approach made it easier to write files, but the data 
structures behind that were more complex, since two list structures 
where needed. GraphEd needed a list of attributes and a list of 
values, whereas we need only a list of key-value pairs.
Another difference is that GraphEd put the graph structure and labels 
into a syntax outside the attributes, where we will combine them, 
once again to have only one data structure.

\section{TEI (SGML) Format}

The format used in TEI [reference] is actually a SGML DTD. It has the 
advantage that SGML provides a powerful standardized framework for it.
However, the SGML syntax is more complex than ours, so parsing is 
more difficult. Our goal is a format that can easily be converted 
into any other format, so a easy parser is essential. Also, we do not 
need a format with the power of SGML here.

\section{VRML Format}

VRML is a format who's syntax is quite similar to the one defined 
here. Basically, it uses a key-value structure. The main difference 
is that a key can have a list values.
The format is extensible.

\section{dot Format}
\TBD{}

\section{Tom Sawyer Software Format}

Tom Sawyer Software, Berkeley makes commercial graph layout and graph 
editor toolkits.  Their file format uses keys in a line started by 
\texttt{//} , followed by a list of values, each on its line:

\begin{verbatim}
// Graph Layout Toolkit
Hierarchical Layout
// minimumSlopePercent
20
// Nodes
2
42
\end{verbatim}

There is no hierarchical structure, although they can be modelled 
with dummy begin/end keys. The format is extensible; new elements can 
be added through a C or C++ interface.


%%% Local Variables: 
%%% mode: latex
%%% TeX-master: "GML"
%%% End: 

% The GML Format
%

\chapter{The GML Format}

This section describes the syntax of a GML file, and how scanner and 
parser for the object tree are constructed. Graphs are of no 
relevance here; the next chapter will show how to extract a graph 
from a GML file.


%
% Syntax
%

\section{Syntax}

Figure \ref{f:GML:Grammar} shows the syntax of GML in BNF
notation. In this format, $x^{+}$ denotes a sequence of one or
more $x$ items, and $x^{*}$ denotes a sequence of zero ore more $x$
items. Characters in `quotes' denote terminal characters, and words in
$<$\emph{angle brackets}$>$ denote nonterminals.


\begin{figure}[htbp]        
  \begin{bnf}{$<$WhiteSpace$>$}

    \Declare{GML}{
      \NonTerm{List}
      }
        
    \Declare{List}{
      \Empty{}
      \Alternative{
        \NonTerm{KeyValue}
        \ZeroOrMore{(\OneOrMore{\NonTerm{WhiteSpace}}\NonTerm{KeyValue})}
        }
      }

    \Declare{KeyValue}{
      \NonTerm{Key}
      \OneOrMore{\NonTerm{WhiteSpace}}
      \NonTerm{Value}
      }

    \Declare{Value}{
      \NonTerm{Integer}
      \Alternative{\NonTerm{Real}}
      \Alternative{\NonTerm{String}}
      \Alternative{\Term{$\left[\right.$} \NonTerm{List} \Term{$\left.\right]$}}
      }
        
    \Declare{Key}{
      \Range{\FromTo{a}{z}\FromTo{A}{Z}}\ZeroOrMore{\Range{\FromTo{a}{z}\FromTo{A}{Z}\FromTo{0}{9}}}
      }
    
    \Declare{Integer}{
      \NonTerm{Sign} \OneOrMore{\NonTerm{Digit}}
      }
    
    \Declare{Real}{
      \NonTerm{Sign}
      \OneOrMore{\NonTerm{Digit}}
      \Term{.}
      \OneOrMore{\NonTerm{Digit}}
      \NonTerm{Mantissa}
      }

    \Declare{String}{
      \Term{"}
      \NonTerm{Instring}
      \Term{"}
      }
        
    \Declare{Sign}{
      \Empty{}
      \Alternative{\Term{+}}
      \Alternative{\Term{-}}
      }
        
    \Declare{Digit}{
      \Range{\FromTo{0}{9}}
      }
        
    \Declare{Mantissa}{
      \Empty{}
      \Alternative{\Term{E} \NonTerm{Sign} \OneOrMore{Digit}}
      \Alternative{\Term{e} \NonTerm{Sign} \OneOrMore{Digit}}
      }
        
    \Declare{Instring}{
      ${\mathrm ASCII} - \{ \Term{\&}, \Term{"} \}$
      \Alternative{\Term{\&} \ZeroOrMore{\NonTerm{character}} \Term{;}}
      }
    
    \Declare{Whitespace}{
      \NonTerm{space}
      \Alternative{\NonTerm{tabulator}}
      \Alternative{\NonTerm{newline}}
      }
          
  \end{bnf}

  \caption{The GML Grammar in BNF Format.}
  \label{f:GML:Grammar}
\end{figure}


\section{Further specifications}

\subsection{ISO 8859 Character Set}

In Figure \ref{f:GML:Grammar}, \emph{Instring} excludes the
characters \verb|"| and \verb|&| characters from a string.  This
is necessary because a \verb|"| inside a string would terminate
the string prematurely.  The \verb|&| character is used by the
ISO 8859 character set to introduce a special character. This
special character starts with an ampersand, is followed by a name
and is terminated by a semicolon.  For example, \verb|&quot;|
inserts \verb|"|, \verb|&amp;| inserts \verb|&|, and
\verb|&auml;| inserts a German `\"{a}'.  For a complete list of
characters, see tables \ref{t:ISO8859-1:basic},
\ref{t:ISO8859-1:special}, \ref{t:ISO8859-1:capital} and
\ref{t:ISO8859-1:lowercase} in section \ref{s:ISO8859}.

We do not allow ISO 8859 characters outside strings, especially
not in keywords. Thus, a sloppy parser might not know the ISO
8859 character set, but just ASCII, and can safely read and write
GML as 7-bit ASCII files. Many applications do in fact not need
to look into the labels, or use only simple labels.

\subsection{Line Length}
The maximum line length in the file format must not exceed
\textbf{254} characters. This ensures that even systems with a
more restrictive line length can cope with a GML file.

\subsection{Key Syntax}
We use a very restricted format for keys, which does not allow
characters such as '\texttt{\_}', '\texttt{\$}' or '\texttt{:}'.
This is because they might not be legal characters
for variables in some interpreted languages.  With this
restriction, keys may be used as identifiers in interpreted
languages.  It also simplifies the syntax a lot.

\subsection{Key Size}
Because of the maximum line length, keys must less than
\textbf{254} characters. However, it is quite convenient to have
a key and a data item on one line, so it is a good idea to have a
key size of less then than 126.

\subsection{Line breaks}
Line breaks may occur anywhere in the file format where white
space is allowed. Line breaks inside strings are line breaks in
the string\footnote{This limits strings to a maximum \emph{line
    width} of 253 characters, which seems reasonable.}.

\subsection{\# Comments}
Any line which starting with \verb|#| (whitespace \emph{before}
the \verb|#| is allowed) is ignored by the parser.  This is a
standard treatment in most UNIX programs.  For example, using the
following as a first line

\begin{alltt}
#!/usr/local/bin/gmlview
\end{alltt}

\noindent specifies that the program \texttt{gmlview} interprets the file.
It is also common to include foreign data (e.g.\ a Postscript
representation of the graph) through that mechanism. Many drawing
programs use the reverse mechanism to insert their data into a
Postscript file.

GML includes also a \texttt{comment} key which adds comments to a
file. However, a parser will read and store these comments, so
they should be reserved for \emph{small} comments. Any comments
inserted with \texttt{\#} should be ignored by the parser.

\subsection{Order of duplicate keys}
\label{s:GML:OrderOfDupliactekeys}
It is perfectly legal to have duplicate keys within the same
list.  For example, an array might be represented as follows:

\begin{quote}
\begin{alltt}
array [
    element [ {\textnormal{\ldots{}}} ]
    element [ {\textnormal{\ldots{}}} ]
    element [ {\textnormal{\ldots{}}} ]
]
\end{alltt}
\end{quote}

\noindent To avoid problems, we require that the order of \emph{duplicate}
keys is \emph{preserved} by the parser. The order of not
duplicate keys does need not be preserved. This is because
programs might not be able to record the exact order of the
attributes in the file. If would also make it more difficult to
add more attribues as the file format grows\footnote{Of course,
  one could require that the new attributes are just appended to
  the old list. However, if a program is written in a modular
  fashion, the attributes will be written by procedures
  \mbox{\texttt{p1}, \texttt{p2}, \ldots, \texttt{pn}} in that
  order. If \texttt{p1} is extended, the new attributes would
  have to be written \emph{after} \texttt{pn}, which would break
  the modularization.}.

\subsection{Unknown Keys}
Any parser which encounters an unknown key should preserve it and
its value, and write them back when the graph is saved into a
file.  Exceptions to this policy are changes in the structure,
e.g.\ deletion of the parent of the unknown objects, and
consistency problems (see \ref{c:GML:Consistency}).

\subsection{Default values}

One important requirement for GML is that an application which
writes a file may omit all ``not interesting'' keys. However, the
following default values should be be assumed for missing
key-value pairs:
\begin{itemize}
  \item \textbf{0} for \emph{Integer} values.
  \item \textbf{0.0} for \emph{Real} values.
  \item \textbf{\texttt{""}} for \emph{String} values.
  \item $\left[\right]$ for \emph{List} values.
\end{itemize}

\noindent This makes sure that files with missing keys are treated
equally by different programs. For example, one could define that
a missing object width is substituted by some default value. But,
what should be used ? Common values are 1, 2, 16, 32, 42 and 64,
or the \emph{current default setting of the program}. However,
especially the last variant is highly dependend on the current
state of the program, and might lead to overlapping objects and
therefore hard to read drawings.

Nevertheless, a program may (and should) implement an option
``substitute defaults for missing values''. Such a clean up
operation\footnote{Graphlet provides such operations in the
  ``Tool'' menu under the keyword ``Clean up''.} which is
available by request is less confusing than the substitution of
system state dependend values.


%
% Graphlet's Parser Implementation
%

\section{Graphlet's Parser Implementation}

It remains to show how to construct an Object tree from a GML file.  
This is done while parsing the second and third rule in the syntax 
definition.

\begin{tabbing}
\emph{List}\texttt{\ ::=\ }\=\texttt{\ \ \ \ }\=\texttt{\ \ \ \ }\=\kill

\emph{List}\texttt{\ ::=\ }
\emph{KeyValue} ( \emph{WhiteSpace}$^+$ \emph{KeyValue})$^{*}$ \\
\> \> \textbf{var} \emph{l}: List(Object); \\
\> \> \emph{l} := emptyList; \\
\> \> \textbf{foreach} \emph{KeyValue} in the list \textbf{do} \\
\> \> \> \emph{o} := new Object(\emph{KeyValue.Key}); \\
\> \> \> \emph{o}.value := \emph{KeyValue.Value}; \\
\> \> \> \textbf{append} \emph{o} to \emph{l}; \\
\> \> \textbf{done} \\
\> \> \textbf{return} l; \\

\\
\emph{Value}\texttt{\ ::=\ }\emph{Integer} $\mid$ \\
\> \> \textbf{return} \emph{Integer}; \\
\>\emph{Real} $\mid$ \\
\> \> \textbf{return} \emph{Real}; \\
\>\emph{String} $\mid$ \\
\> \> \textbf{return} \emph{String}; \\
\>\verb|[| \emph{List} \verb|]| \\
\> \> \textbf{return} \emph{List};
\end{tabbing}


\section{How to represent common data structures}

As earlier said, GML is by no means restricted to graphs. In
fact, all common data types can be represented in GML. This
section is a cookbook for designing data structure
representations in GML.

\subsection{Boolean}
Boolean values can be represented by Integers. \texttt{false} is
represented by 0, \texttt{true} is represented by any other
number.

\begin{notes}
  \item We decided not to implement a separate datatype for
  boolean values because this would only complicate the parser.
\end{notes}


\subsection{Large numbers}
\label{s:GML:LargeNumbers}
Large integer or floating point data types can be represented as
Strings.  The string is the standard ASCII representation of the
value.

\begin{notes}
  \item A more compact representation for large integer values
  could be obtained by coding the bits of the number in the same
  fashion as the \texttt{uuencode} program which is available on
  UNIX systems. However, this would assume that the layout of the
  bits in memory is fixed (which is \emph{not} the case), and
  would also complicate the parser. It is also not clear how many
  applications would need this.
  % except TL
  \item If space is an issue, An application may encode very
  large integer numbers as a list if integers in a $2^{15}$- or
  $2^{31}$-adic number system.
\end{notes}

\subsection{Bitset}
\label{s:GML:Bitset}
A bitset should be represented as a string of 0 and 1 characters.

\begin{notes}
  \item The same as for large numbers (section
  \ref{s:GML:LargeNumbers}) applies here, too.
\end{notes}


\subsection{Records}

A record data type

\begin{quote}
\begin{alltt}
name: record
    a: type1
    b: type2
    \ldots{}
end
\end{alltt}
\end{quote}

\noindent can be represented in GML as follows:

\begin{quote}
\begin{alltt}
name [
    a \ttcomment{value\_of\_a}
    b \ttcomment{value\_of\_b}
    \ttcomment{\ldots{}}
]
\end{alltt}
\end{quote}

\noindent In place of \emph{value\_of\_field}, insert the value of
the corresponding field of the record. This can either be an
integer, a real or a string value or a list.


\subsection{Lists, Sets, Arrays}

A list of data type

\begin{quote}
\begin{alltt}
name: List(SomeType)
\end{alltt}
\end{quote}

\noindent can be represented in GML as follows:

\begin{quote}
\begin{alltt}
name [
  obj \ttcomment{value\_of\_first\_element}
  obj \ttcomment{value\_of\_second\_element}
  \ldots{}
]
\end{alltt}
\end{quote}

\noindent where \texttt{obj} should be replaced by a suitable name for the
elements of the list. Section \ref{s:GML:OrderOfDupliactekeys}
specifies that a parser must preserve the order of the
\texttt{obj} attributes when it reads the program.

Sets and arrays are represented with exactly the same schema. In
the case of arrays, it might be neccessary to use a list for
\emph{value\_of\_field} and add an \texttt{id} key to the value.
This \texttt{id} represents the index of the array element.
Associative arrays might use a \texttt{name} key or some more
complex structure.


\subsection{Name clashes}

The freedom of adding new keys bears the problem of name clashes.
To avoid this problem as much as possible, any application should
put its private information in a object of type list who's
private key represents the name of the application.  For example,
the Graphlet system will insert all private information into a an
object with key named \texttt{Graphlet}\footnote{Of course, this
  is still not 100\% save, but as close as we can get with a
  reasonable effort. HTML uses the same approach and works still
  fine after all these years.}.


%
% Extracting Graphs from GML files
%

\chapter{Extracting Graphs from GML files}

Up to this point, GML was not related to graphs in any way.  In
fact, GML is designed so that it can map any data structure onto
an ASCII file.  To extract a graph from a GML file, we parse the
file and extract the list of Object structures.  Then, we run
through the objects and extract the graph structure from them:

\begin{program}
\textbf{var} \emph{objects}: List(Object); \\
\\
\emph{objects} := parse (file); \\
\textbf{foreach} $o$ \textbf{in} \emph{objects} \textbf{where} $o$.key = ``Graph'' \textbf{do} \\
\> \emph{g} := \textbf{new} graph; \\
\> \textbf{foreach} $\bar{o}$ \textbf{in} $o$.value \textbf{where} $\bar{o}$.key = ``Node'' \textbf{do} \\
\> \> $n$ := \textbf{new} node ($g$); \\
\> \> $n$.attributes := $\bar{o}$.value; \\
\> \> \textbf{remove} $\bar{o}$ \textbf{from} $o$.value; \\
\> \textbf{done} \\
\> \textbf{foreach} $\bar{o}$ \textbf{in} $o$.value \textbf{where} $\bar{o}$.key = ``Edge'' \textbf{do} \\
\> \> $e$ := \textbf{new} edge ($\bar{o}$.source.value, $\bar{o}$.target.value); \\
\> \> $e$.attributes := $\bar{o}$.value; \\
\> \> \textbf{remove} $\bar{o}$ \textbf{from} $o$; \\
\> \textbf{done} \\
\> $g$.attributes := $o$.value; \\
\> \textbf{remove} $o$ \textbf{from} \emph{objects}; \\
\textbf{done} \\
\end{program}

\noindent To write a graph to a GML file, use the following schema:

\begin{program}
\textbf{procedure} print ($g$: Graph); \\
\textbf{begin} \\
\> \textbf{print} ``\verb|graph [|''; \\
\> \textbf{print} ($g$.attributes); \\
\> \textbf{foreach} $n$ \textbf{in} g.nodes \textbf{do} \\
\> \> \textbf{print} ``\verb|node [|''; \\
\> \> \textbf{print} ($n$.attributes); \\
\> \> \textbf{print} \verb|]|''; \\
\> \textbf{done} \\
\> \textbf{foreach} $e$ \textbf{in} $g$.edges \textbf{do} \\
\> \> \textbf{print} ``\verb|edge [|''; \\
\> \> \textbf{print} $e$.attributes; \\
\> \> \textbf{print} ``\verb|]|''; \\
\> \textbf{done} \\
\> \textbf{print} ``\verb|]|''; \\
\textbf{end} \\
\\
\textbf{procedure} print (\emph{objects}: List(Object)) \\
\textbf{begin} \\
\> \textbf{foreach} $o$ \textbf{in} \emph{objects} \textbf{do} \\
\> \> \textbf{print} $o$.key; \\
\> \> \textbf{case} type($o$) \textbf{of} \\
\> \> \> Integer : \\
\> \> \> \> \textbf{print} $o$.value; \\
\> \> \> Real : \\
\> \> \> \> \textbf{print} $o$.value; \\
\> \> \> String : \\
\> \> \> \> \textbf{print} $o$.value; \ttcomment{must ensure line
 length $<$ 255} \\
\> \> \> List : \\
\> \> \> \> \textbf{print} ``\verb|[|'' \\
\> \> \> \> \textbf{print} $o$.value; \\
\> \> \> \> \textbf{print} ``\verb|]|''; \\
\> \> \textbf{end} \\
\> \textbf{done} \\
\textbf{end} \\
\end{program}

\noindent In the above program, we assume that a statement \textbf{print
  $s$} writes $s$ with the following properties:

\begin{itemize}
  \item $s$ is surrounded by quotes.
  \item The output is adjusted to a maximum line length of 254 characters, 
  inclusive quotes.
  \item All characters are properly translated into the ISO 8859
  character set. Especially, the \verb|&| and \verb|"| characters
  are replaced by \verb|&amp;| and \verb|&quot;|.
\end{itemize}
  
It is a good idea to save all objects in the file and print them
out again, unless they are unsafe and changes to the graph have
been made (see also chapter \ref{c:GML:Consistency}).  This will
make sure that valuable information supplied by other programs
wont get lost.
  

%
% Consistency
%
  

\chapter{Consistency: Safe and Unsafe Objects}
\label{c:GML:Consistency}

With GML, we can specify objects that depend on other objects.
That is, changing one object forces other objects to be updated.
There are many examples for such behaviour:

\begin{description}
  \item[Edge coordinates] Edge coordinates must be updated when
  the endnodes of the edge move.  Technically, if a node is
  removed, all adjacent edges must be removed.
  
  \item[Label coordinates] The same as above applies to label
  coordinates.
 
  \item[Graph Theory] Information on graph theoretical
  properties, such as planarity, is often needed by applications
  and should therefore be added to a graph. This may save costly
  recomputation (if an appropriate algorithm is available at
  all). This information must be updated whenever the structure
  of the graph changes.
  
  \item[Maximum and Minimum Coordinates] Some applications need
  information on the maximum or minimum x and y coordinates.
  They must be updated whenever a node or edge changes
  \emph{position} or \emph{size}.
  
  \item[Size of the graph] Information on the size of the graph,
  that is the number of nodes and/or edges is also an often
  found information in graph file formats.  However, this
  information must be updated whenever the structure of the graph
  changes.
\end{description}

In some cases, it is quite obvious how to handle updates.  On the
other hand, in the case of Graph theoretic properties, it might
be difficult and time costly to update, if at all feasible.

There are many solutions to this problem. Besides omitting all
potentially dangerous keys, a simple solution would be to define
consistency conditions for a few known objects and forbid any
other consistency violating objects.  This would especially mean
that graph theoretical properties, which might be the result of
several hours computing, are ruled out.

The most complex solution would be to give a precise
specification for the dependencies of each object, and force
updates. However, this would lead to a very complex file formant,
and would be difficult to implement.

Therefore, we settle for the following solution: It is legal to
add objects which become inconsistent after changes, but a a hint
must be given that the key bears a potential consistency problem.

\begin{definition}[Safe and Unsafe Objects]
  We call any object which must be updated or removed upon a
  change unsafe.  We discriminate safe and unsafe objects by
  their key:
  \begin{itemize}
    \item An object who's key starts with a \textbf{lower case} letter
    starts is \emph{safe}.
    \item An object who's key starts with a \textbf{capital letter} starts
    is \emph{unsafe}.
  \end{itemize}
  Any program that changes (a) the topoligical structure of the
  graph or (b) an attribute must delete all objects it cannot
  update properly.
\end{definition}





%%% Local Variables: 
%%% mode: latex
%%% TeX-master: "GML"
%%% End: 

\appendix
%%%%%%%%%%%%%%%%%%%%%%%%%%%%%%%%%%%%%%%%%%
%
% Predefined Keys
%
%%%%%%%%%%%%%%%%%%%%%%%%%%%%%%%%%%%%%%%%%%

\chapter{Predefined Keys}

This section lists predefined keys. The design of GML leaves
applications any freedom to define their own keys. However, in
order to maintain compatibility with other applications, standard
keys should be used as much as possible.  The entries in the
lists below will be as follows:

\begin{quote}
  \texttt{Key}
  \emph{Type}
  \hfill
  \makebox[5cm][l]{\texttt{Context}}\qquad\textbf{Save/Unsafe}
\end{quote}

\noindent where

\begin{description}

  \item[\texttt{Key}] is the (name of the) GML Key.

  \item[\emph{Type}] is the type of the value.
  We use the following types:
  \begin{itemize}
    \item \texttt{Integer} is an integer value.
    \item \texttt{Real} is a floating point value.
    \item \texttt{String} is a string value.
    \item \texttt{List} is a list value.
    \emph{true}.
  \end{itemize}
  
  \item[\texttt{Context}] specifies a path whichmust be the
  prefix for the key:
  \begin{itemize}
    \item If the context is ``\texttt{.a.b.c}'' and the key is
    ``\texttt{x}'', then this key must be on a path
    ``\texttt{.a.b.c.x}''.
    % Hint for blondes
    If the type is not \emph{List}, then \texttt{x} is obviously
    a leaf in the object tree.
    \item If the context is ``\texttt{.}'', then the key may
    occur everywhere.
    \item If the context is empty, then this key is defined only
    if it occurs at the top level
  \end{itemize}
  
  
  \item[\textbf{Safe}, \textbf{Unsafe}] specifies whether this
  object should generally be regarded as safe (see Chapter
  \ref{c:GML:Consistency}).

\end{description}


\section{Global Attributes}
\label{s:GML:GlobalAttributes}

The following attributes can be used with all objects.


\begin{GMLAttributes}
  
  \GMLAttr{id}{Integer}{.}
  An id is an identifier for an object.
  The values of id objects do not need to be unique throughout a
  file; this is defined by the application.

  \GMLAttr{label}{String}{.}  
  Defines the label of an object. A graphical browser should use
  this attribute to annotate the object with a text.
  
  \GMLAttr{comment}{String}{.} Defines a comment. Note that such
  a comment will be read \emph{and} saved. This attribute is
  \textbf{safe}, so handle with care. There is no unsafe version
  of \texttt{comment} since this would not be helpful as comment
  is too general.
  
  \GMLAttr[unsafe]{Creator}{String}{.} The program which created
  this object. This attribute is unsafe since every program
  should register its own name here, or forget about it.
  
  \GMLAttr{name}{String}{.} Defines the name of an object. The
  name is a textual alternative to the \texttt{id} attribute.

\end{GMLAttributes}


\section{Top Level Attributes}
\label{s:GML:TopLevelAttributes}

The following attributes are only defined at the top level of a
file, that is they may have no prefix.

\begin{GMLAttributes}
  
  \GMLAttr{graph}{List}{} Defines a graph. In the current
  version of GML, the top object is almost always a graph.
  
  \GMLAttr[unsafe]{Version}{Integer}{} Describes the GML version of this
  file. Current value is 1. Files with an unknown version number
  should be rejected.

  \GMLAttr[unsafe]{Creator}{String}{} The name of the program which
  created this file.

\end{GMLAttributes}

The fact that these attributes are only defined at the top level
does not mean that they can not be defined at other locations.
For example, given a proper semantics, graphs can be defined
inside graphs.


%
% Graphs
%

\section{Graphs}
\label{s:GML:Graph}


%
% Graph Attributes
%

\subsection{Graph Attributes}
\label{s:GML:GraphAttributes}

\begin{GMLAttributes}
  
  \GMLAttr{directed}{B}{.graph} Defines whether a graph is directed
  (0) or undirected (1). Default is undirected.
  
  \GMLAttr{node}{L}{.graph} Defines a node. Each node should have
  an id attribute, which must be unique within all node.id
  attributes of a graph object.

  \GMLAttr{edge}{L}{.graph}
  Defines a new edge..edge.source and.edge.target reference the
  endnodes of an edge

\end{GMLAttributes}


%
% Node Attributes
%

\subsection{Node Attributes}
\label{s:GML:NodeAttributes}

A node is usually identified by its id attribute, and often has a 
label attributes Its graphical properties are described by the 
graphics attribute.

\begin{GMLAttributes}

  \GMLAttr{id}{I}{.graph.node} Defines an identification number
  for the node. All \texttt{id} numbers within a graph must be
  unique.
  
  \GMLAttr{edgeAnchor}{S}{.graph.node} Defines how the edges are
  attached to the node.

\end{GMLAttributes}


%
% Graphics Attributes
%

\section{Graphics Attributes}
\label{s:GML:GraphicsAttributes}

Generally, the graphics attributes in GML are modelled after the
graphics in Tk.


\begin{GMLAttributes}
  
  \GMLAttr{x}{R}{.graphics} Defines the \emph{x} coordinate
  of the center of the bounding rectangle of the object.
  
  \GMLAttr{y}{R}{.graphics} Defines the \emph{y} coordinate
  of the center of the bounding rectangle of the object.
  
  \GMLAttr{z}{R}{.graphics.center} Defines the \emph{z}
  coordinate\NYI{} of the center of the bounding rectangle of
  the object.
 
  \GMLAttr{w}{R}{.graphics} Define the width of the bounding
  box of the node. If omitted, the width is 0.

  \GMLAttr{h}{R}{.graphics} Define the height of the bounding
  box of the node. If omitted, the height is 0.
  
  \GMLAttr{d}{R}{.graphics} Define the depth\NYI[foot] of
  the bounding box of the node. If omitted, the depth is 0.

  % Add more dimensions according to your degree of enlightment.
  
  \GMLAttr{type}{S}{.graphics} Defines the graphical object.
  Values for \texttt{type} are \texttt{arc}, \texttt{bitmap},
  \texttt{image}, \texttt{line}, \texttt{oval}, \texttt{polygon},
  \texttt{rectangle} and \texttt{text}.

  \begin{notes}
    \item The current implementation of Graphlet assumes that
    edges always have the type \texttt{line}.
    \item The current implementation of Graphlet assumes that
    labels always have the type \texttt{text}.
  \end{notes}


  \GMLAttr{image}{S}{.graphics} The value of \texttt{image} is
  the name of an image file which is used to draw an object.  If
  no \texttt{w}, \texttt{h} and \texttt{d} attributes are given,
  the application my use the images dimensions to scale the
  graphics.  Otherwise, the image is scaled to fit into the given
  dimensions.

  \begin{notes}
    
    \item The actual image capabilities may depend on the
    graphics subsystem used by the program\footnote{Graphlet can
      read GIF and JPEG files through Tk, but cannot scale
      them in the current version.}. For example, not
    all systems can display pictures in JPEG format, or scale
    them.  Programs should replace the image by a dummy object if
    sufficient graphics capabilities are not available.
    
    \item The corresponding\texttt{image} attribute in
    \GraphScript{} and \GraphScript's C++ interface is slightly
    different as this attribute needs a Tcl/Tk \texttt{image}
    object instead of a filename. We are using a different
    approach here to be more portable.
    
    \item Images are always stored in separate files. This is
    because application programmer interfaces for graphics file
    formats almost always support reading from a file, while
    converting from a string in a file is usually not supported.

  \end{notes}
  
  \GMLAttr{bitmap}{S}{.graphics} The value of \texttt{bitmap} is
  the name of a bitmap file which is used to draw an object. A
  \texttt{bitmap} differs from an \texttt{image} in that the
  bitmap has only a foreground and a background color, while
  images may have their own arbitrary color table.
  
  \GMLAttr{point}{L}{.graphics.Line} Defines a point of the
  polyline that is used to draw the edge.  A straight line edge
  does not need to specify points. If points are specified, they
  must include the end points of the edge.
  
  The following example shows how to use the \texttt{Line}
  attribute:
  \begin{quote}
    \begin{small}
\begin{verbatim}
graphics [
    Line [
        point [
            x 10.0
            y 10.0
        ]
        point [
            x 100
            y 20.0
        ]
        point [
            x 20.0
            y 20.0
        ]
    ]
]
\end{verbatim}
    \end{small}
  \end{quote}
  
  \noindent The example specifies a line with three points. In reality,
  this corresponds to an edge with one bend (the first and the
  last \texttt{point} entries are the endpoints of the edge).

  \begin{notes}
    \item If the \texttt{Line} attribute of an edge is omitted,
    an application should substitute a straigt line between the
    endpoints of the edge.
    \item \texttt{Line} us \emph{unsafe} because the coordinates
    become invalid as soon as the endpoints are moved.
  \end{notes}
  
  \GMLAttr{point}{L}{.graphics.Line} Defines a point of the
  polyline that is used to draw the edge.  A straight line edge
  does not need to specify points. If points are specified, they
  \emph{must} include the end points of the edge.
  
  \GMLAttr{point.x}{R}{.graphics.Line} Define the \emph{z}
  coordinate of a point.
  
  \GMLAttr{point.y}{R}{.graphics.Line} Define the \emph{z}
  coordinate of a point.
  
  \GMLAttr{point.z}{R}{.graphics.Line} Define the \emph{z}
  coordinate of a point.
  
  \GMLAttr{width}{R}{.edge.graphics} Sets the line width (in
  pixels). If nothing is specified, 0.0 should be assumed.
  
  \GMLAttr{stipple}{S}{.edge.graphics} Defines a stipple pattern
  to draw the line (if the drawing system supports that).

\end{GMLAttributes}

\noindent \emph{This section is subject to be extended.}


\subsection{Edge Attributes}
\label{s:GML:EdgeAttributes}

Each edge must have attributes source and target to specify its
endnodes.In an undirected graph, there is no graph theoretical
distinction between source and target, but the coordinates of the
drawing might impose a direction on the edge.

\begin{GMLAttributes}

  \GMLAttr{source}{I}{.edge}  
  In a directed graph, \texttt{source} defines the source node of
  an edge.  In an undirected graph, source defines one of the
  endpoints of an edge.

  \GMLAttr{target}{I}{.edge}
  In a directed graph, \texttt{target} defines the source node of
  an edge.  In an undirected graph, target defines one of the
  endpoints of an edge.

\end{GMLAttributes}



%%% Local Variables: 
%%% mode: latex
%%% TeX-master: "GML.tex"
%%% End: 

%%%%%%%%%%%%%%%%%%%%%%%%%%%%%%%%%%%%%%%%%%
%
% ISO 8859 Characters
%
%%%%%%%%%%%%%%%%%%%%%%%%%%%%%%%%%%%%%%%%%%


\chapter{ISO 8859 Characters}
\label{s:ISO8859}

Tables\footnote{This is an adapted version of Martin Ramsch's ISO
  8859 tables, available on the world wide web at
  \texttt{http://www.uni-passau.de/~ramsch/iso8859-1.html}. The
  author has granted us the permission to modify and use his
  table} \ref{t:ISO8859-1:basic}, \ref{t:ISO8859-1:special},
\ref{t:ISO8859-1:capital} and \ref{t:ISO8859-1:lowercase} present
an overview of the ISO 8859-1 standard character set.

%small sharp s, German (sz ligature)\=
%\textbf{Character}\=
%\verb|&#255;|\=
%\kill

%\textbf{Description} \>
%\textbf{Character} \>
%\textbf{Code} \>
%\textbf{Entity name} \\

\begin{table}[htbp]
  \begin{center}
    \leavevmode

    \begin{tabular}{lc>{\ttfamily}c>{\ttfamily}c}
      \textbf{Description} &
      \textbf{Character} &
      \textbf{Code} &
      \textbf{Entity Name} \\
      \hline
      quotation mark             & "   & \&\#34; & \&amp; \\
      ampersand                  & \&  & \&\#38; & \&amp; \\
      less-than sign             & $<$ & \&\#60; & \&lt; \\
      greater-than sign          & $>$ & \&\#62; & \&gt; \\
    \end{tabular}
    
    \caption{ISO 8859-1 Characters, basic}
    \label{t:ISO8859-1:basic}
  \end{center}
\end{table}

\begin{table}[htbp]
  \begin{center}
    \leavevmode

    \begin{tabular}{lc>{\ttfamily}c>{\ttfamily}c}
      \textbf{Description} &
      \textbf{Character} &
      \textbf{Code} &
      \textbf{Entity Name} \\
      \hline
      non-breaking space                & \verb*| |  & \&\#160; & \&nbsp; \\
      inverted exclamation              & !`         & \&\#161; & \&iexcl; \\
      cent sign                         & ???        & \&\#162; & \&cent; \\
      pound sterling                    & \pounds    & \&\#163; & \&pound; \\
      \hline
      general currency sign             & ???        & \&\#164; & \&curren; \\
      yen sign                          & ???        & \&\#165; & \&yen; \\
      broken vertical bar               & $\mid$     & \&\#166; & \&brvbar; \\
      section sign                      & \S         & \&\#167; & \&sect; \\
      \hline
      umlaut (dieresis)                 & \"{ }      & \&\#168; & \&uml; \\
      copyright                         & \copyright & \&\#169; & \&copy; \\
      feminine ordinal                  & ???        & \&\#170; & \&ordf; \\
      left angle quote, guillemotleft   & ???        & \&\#171; & \&laquo; \\
      \hline
      not sign                          & ???        & \&\#172; & \&not; \\
      soft hyphen                       & ???        & \&\#173; & \&shy; \\
      registered trademark              & ???        & \&\#174; & \&reg; \\
      macron accent                     & ???        & \&\#175; & \&macr; \\
      \hline
      degree sign                       & ???        & \&\#176; & \&deg; \\
      plus or minus                     & ???        & \&\#177; & \&plusmn; \\
      superscript two                   & ???        & \&\#178; & \&sup2; \\
      superscript three                 & ???        & \&\#179; & \&sup3; \\
      \hline
      acute accent                      & ???        & \&\#180; & \&acute; \\
      micro sign                        & ???        & \&\#181; & \&micro; \\
      paragraph sign                    & ???        & \&\#182; & \&para; \\
      middle dot                        & ???        & \&\#183; & \&middot; \\
      \hline
      cedilla                           & ???        & \&\#184; & \&cedil; \\
      superscript one                   & ???        & \&\#185; & \&sup1; \\
      masculine ordinal                 & ???        & \&\#186; & \&ordm; \\
      right angle quote, guillemotright & ???        & \&\#187; & \&raquo; \\
      \hline
      fraction one-fourth               & ???        & \&\#188; & \&frac14; \\
      fraction one-half                 & ???        & \&\#189; & \&frac12; \\
      fraction three-fourths            & ???        & \&\#190; & \&frac34; \\
      inverted question mark            & ???        & \&\#191; & \&iquest; \\
      \hline
      division sign                       & ??? & \&\#247; & \&divide; \\
    \end{tabular}
    
    \caption{ISO 8859-1 Characters, Special and Mathematical Characters}
    \label{t:ISO8859-1:special}
  \end{center}
\end{table}

\begin{table}[htbp]
  \begin{center}
    \leavevmode

    \begin{tabular}{lc>{\ttfamily}c>{\ttfamily}c}
      \textbf{Description} &
      \textbf{Character} &
      \textbf{Code} &
      \textbf{Entity Name} \\
      \hline
      capital A, grave accent            & ??? & \&\#192; & \&Agrave; \\
      capital A, acute accent            & ??? & \&\#193; & \&Aacute; \\
      capital A, circumflex accent       & ??? & \&\#194; & \&Acirc; \\
      capital A, tilde                   & ???    & \&\#195; & \&Atilde; \\
      \hline
      capital A, dieresis or umlaut mark & ??? & \&\#196; & \&Auml; \\
      capital A, ring                    & ??? & \&\#197; & \&Aring; \\
      capital AE diphthong (ligature)    & ??? & \&\#198; & \&AElig; \\
      capital C, cedilla                 & ??? & \&\#199; & \&Ccedil; \\
      \hline
      capital E, grave accent            & ??? & \&\#200; & \&Egrave; \\
      capital E, acute accent            & ??? & \&\#201; & \&Eacute; \\
      capital E, circumflex accent       & ??? & \&\#202; & \&Ecirc; \\
      capital E, dieresis or umlaut mark & ??? & \&\#203; & \&Euml; \\
      \hline
      capital I, grave accent            & ??? & \&\#204; & \&Igrave; \\
      capital I, acute accent            & ??? & \&\#205; & \&Iacute; \\
      capital I, circumflex accent       & ??? & \&\#206; & \&Icirc; \\
      capital I, dieresis or umlaut mark & ??? & \&\#207; & \&Iuml; \\
      \hline
      capital Eth, Icelandic             & ??? & \&\#208; & \&ETH; \\
      capital N, tilde                   & ??? & \&\#209; & \&Ntilde; \\
      capital O, grave accent            & ??? & \&\#210; & \&Ograve; \\
      capital O, acute accent            & ??? & \&\#211; & \&Oacute; \\
      \hline
      capital O, circumflex accent       & ??? & \&\#212; & \&Ocirc; \\
      capital O, tilde                   & ??? & \&\#213; & \&Otilde; \\
      capital O, dieresis or umlaut mark & ??? & \&\#214; & \&Ouml; \\
      multiply sign                      & ??? & \&\#215; & \&times; \\
      \hline
      capital O, slash                   & ??? & \&\#216; & \&Oslash; \\
      capital U, grave accent            & ??? & \&\#217; & \&Ugrave; \\
      capital U, acute accent            & ??? & \&\#218; & \&Uacute; \\
      capital U, circumflex accent       & ??? & \&\#219; & \&Ucirc; \\
      \hline
      capital U, dieresis or umlaut mark & ??? & \&\#220; & \&Uuml; \\
      capital Y, acute accent            & ??? & \&\#221; & \&Yacute; \\
      capital THORN, Icelandic           & ??? & \&\#222; & \&THORN; \\
    \end{tabular}
    \caption{ISO 8859-1 Characters, Capital Letters}
    \label{t:ISO8859-1:capital}
  \end{center}
\end{table}

\begin{table}[htbp]
  \begin{center}
    \leavevmode


    \begin{tabular}{lc>{\ttfamily}c>{\ttfamily}c}
      \textbf{Description} &
      \textbf{Character} &
      \textbf{Code} &
      \textbf{Entity Name} \\
      \hline
      small sharp s, German (sz ligature) & ??? & \&\#223; & \&szlig; \\
      small a, grave accent               & ??? & \&\#224; & \&agrave; \\
      small a, acute accent               & ??? & \&\#225; & \&aacute; \\
      small a, circumflex accent          & ??? & \&\#226; & \&acirc; \\
      \hline
      small a, tilde                      & ??? & \&\#227; & \&atilde; \\
      small a, dieresis or umlaut mark    & ??? & \&\#228; & \&auml; \\
      small a, ring                       & ??? & \&\#229; & \&aring; \\
      small ae diphthong (ligature)       & ??? & \&\#230; & \&aelig; \\
      \hline
      small c, cedilla                    & ??? & \&\#231; & \&ccedil; \\
      small e, grave accent               & ??? & \&\#232; & \&egrave; \\
      small e, acute accent               & ??? & \&\#233; & \&eacute; \\
      small e, circumflex accent          & ??? & \&\#234; & \&ecirc; \\
      \hline
      small e, dieresis or umlaut mark    & ??? & \&\#235; & \&euml; \\
      small i, grave accent               & ??? & \&\#236; & \&igrave; \\
      small i, acute accent               & ??? & \&\#237; & \&iacute; \\
      small i, circumflex accent          & ??? & \&\#238; & \&icirc; \\
      \hline
      small i, dieresis or umlaut mark    & ??? & \&\#239; & \&iuml; \\
      small eth, Icelandic                & ??? & \&\#240; & \&eth; \\
      small n, tilde                      & ??? & \&\#241; & \&ntilde; \\
      small o, grave accent               & ??? & \&\#242; & \&ograve; \\
      \hline
      small o, acute accent               & ??? & \&\#243; & \&oacute; \\
      small o, circumflex accent          & ??? & \&\#244; & \&ocirc; \\
      small o, tilde                      & ??? & \&\#245; & \&otilde; \\
      small o, dieresis or umlaut mark    & ??? & \&\#246; & \&ouml; \\
      \hline
      small o, slash                      & ??? & \&\#248; & \&oslash; \\
      small u, grave accent               & ??? & \&\#249; & \&ugrave; \\
      small u, acute accent               & ??? & \&\#250; & \&uacute; \\
      small u, circumflex accent          & ??? & \&\#251; & \&ucirc; \\
      \hline
      small u, dieresis or umlaut mark    & ??? & \&\#252; & \&uuml; \\
      small y, acute accent               & ??? & \&\#253; & \&yacute; \\
      small thorn, Icelandic              & ??? & \&\#254; & \&thorn; \\
      small y, dieresis or umlaut mark    & ??? & \&\#255; & \&yuml; \\
    \end{tabular}
    \caption{ISO 8859-1 Characters, Lowercase Letters}
    \label{t:ISO8859-1:lowercase}
  \end{center}
\end{table}


\begin{notes}
  \item Since LaTeX does not use the 8859-1 character set, the
  above table is only an approximation.
  \item Entries marked ``???'' in the above table will be filled
  in the final version of this document.
\end{notes}


%%% Local Variables: 
%%% mode: latex
%%% TeX-master: "GML.tex"
%%% End: 


\end{document}

%%% Local Variables: 
%%% mode: latex
%%% End: 
