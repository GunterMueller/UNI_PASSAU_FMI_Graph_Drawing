% --------------------------------------------------------------------------
%
% Programming the NEI
%
% --------------------------------------------------------------------------

%\documentclass[12pt,a4paper]{report}
%\usepackage{epsf,amstex,epic}

\documentclass[twoside,fleqn]{report}
\usepackage{Graphlet,epic}

%\usepackage{epsf,amstex,epic}

\begin{document}
\setlength{\parindent}{0pt} 
\sloppy

% --------------------------------------------------------------------------
% Begin document
% --------------------------------------------------------------------------


% This file is generated automatically
\newcommand{\GraphletVersion}{1}

\tableofcontents

\chapter{Node Edge Interface}

\Graphlet{} has a new Node Edge Interface (NEI).
This section describes how a programmer can use it.

% --------------------------------------------------------------------------
% The idea for the edgeanchors
% --------------------------------------------------------------------------

\section{Basic Concept}
\label{chap:ea_introduction}

The basic idea for our NEI are edgeanchors.
An edgeanchor is a theoretical endpoint of the edge. It is placed
inside the rectangle surrounding each node.
There are two edgeanchors for every edge: One for the source, the other 
for the target.
The edge is clipped at the border of the node each time the node is 
redrawn.\\

The edgeanchors don't have absolute coordinates. They are set 
relativly to the node center.
The distances to the nodecenter are given as factors of the 
height and width of the node.
Thus, one edgeanchor for source or target consists of two real values 
between -1 and 1. From now on we will call these two values 
\emph{delta$_x$} and \emph{delta$_y$}.\\


An edgeanchor can either be set to an certain value by the user 
or to a default function, which is executed each time an edge has to be 
redrawn.
%The advantage of this idea is the independence from 
%changes of the size, style and position of a node.\\


%We have a set of predefined functions to manipulate the edgeanchors.
%You can apply them once to an edge to change the edgeanchor. 

%We also allow to take some of this functions as a kind of 'hook up'. 
%This means, that the function is executed each time an edge has 
%to be redrawn. In this case the function changes the edgeanchor each time
%the edge is redrawn.
%The actual hook up function is called the default function.

\textbf{Note:} 
%We have default functions for nodes and edges. 
It is possible to specify default functions for nodes (they influence the edgeanchors
of all adjacent edges) and edges.
The default function for an edge has always priority over the default
function for adjacent nodes.

% --------------------------------------------------------------------------
% If I have the nodeposition and the edgeanchor: 
% How can I get the edgerouting?
% --------------------------------------------------------------------------

\section{The Edge Routing for a Given Edgeanchor}

Suppose you have a node at (
\emph{x}, 
\emph{y}) with width 
\emph{w} and height 
\emph{h}.
The edgeanchor is set to (\emph{delta$_x$}, \emph{delta$_y$}).

% Picture of example (nei_example.fig)

%\setlength{\unitlength}{0.00041667in}
\begin{center}
\setlength{\unitlength}{0.00026667in}
%
\begingroup\makeatletter\ifx\SetFigFont\undefined
% extract first six characters in \fmtname
\def\x#1#2#3#4#5#6#7\relax{\def\x{#1#2#3#4#5#6}}%
\expandafter\x\fmtname xxxxxx\relax \def\y{splain}%
\ifx\x\y   % LaTeX or SliTeX?
\gdef\SetFigFont#1#2#3{%
  \ifnum #1<17\tiny\else \ifnum #1<20\small\else
  \ifnum #1<24\normalsize\else \ifnum #1<29\large\else
  \ifnum #1<34\Large\else \ifnum #1<41\LARGE\else
     \huge\fi\fi\fi\fi\fi\fi
  \csname #3\endcsname}%
\else
\gdef\SetFigFont#1#2#3{\begingroup
  \count@#1\relax \ifnum 25<\count@\count@25\fi
  \def\x{\endgroup\@setsize\SetFigFont{#2pt}}%
  \expandafter\x
    \csname \romannumeral\the\count@ pt\expandafter\endcsname
    \csname @\romannumeral\the\count@ pt\endcsname
  \csname #3\endcsname}%
\fi
\fi\endgroup
{\renewcommand{\dashlinestretch}{30}
\begin{picture}(9723,8449)(0,-10)
\drawline(1587,3847)(2037,3397)
\drawline(1587,3397)(2037,3847)
\dottedline{45}(612,8422)(612,1222)
\drawline(1212,7222)(8412,7222)(8412,2422)
        (1212,2422)(1212,7222)
\drawline(4587,4822)(5037,4822)
\drawline(4812,5047)(4812,4597)
\dottedline{45}(1812,8422)(1812,1222)
\dottedline{45}(2412,8422)(2412,1222)
\dottedline{45}(3012,8422)(3012,1222)
\dottedline{45}(3612,8422)(3612,1222)
\dottedline{45}(4212,8422)(4212,1222)
\dottedline{45}(5412,8422)(5412,1222)
\dottedline{45}(6012,8422)(6012,1222)
\dottedline{45}(7212,8422)(7212,1222)
\dottedline{45}(7812,8422)(7812,1222)
\dottedline{45}(8412,8422)(8412,1222)
\dottedline{45}(9012,8422)(9012,1222)
\dottedline{45}(12,7822)(9612,7822)
\dottedline{45}(12,6622)(9612,6622)
\dottedline{45}(12,6022)(9612,6022)
\dottedline{45}(12,5422)(9612,5422)
\dottedline{45}(12,4822)(9612,4822)
\dottedline{45}(12,4222)(9612,4222)
\dottedline{45}(12,3622)(9612,3622)
\dottedline{45}(12,3022)(9612,3022)
\dottedline{45}(12,2422)(9612,2422)
\dottedline{45}(12,1822)(9612,1822)
\dottedline{45}(4812,8422)(4812,1222)
\dottedline{45}(6612,8422)(6612,1222)
\drawline(3687,22)(1812,3622)
\drawline(1976.079,3436.857)(1812.000,3622.000)(1869.649,3381.425)
\dottedline{45}(12,7222)(9612,7222)
\dottedline{45}(1212,8422)(1212,1222)
\drawline(4812,2197)    (4925.282,2112.097)
        (5024.111,2040.257)
        (5110.245,1980.380)
        (5185.440,1931.369)
        (5310.049,1861.546)
        (5412.000,1822.000)

\drawline(5412,1822)    (5532.532,1811.874)
        (5674.325,1833.638)
        (5751.189,1851.948)
        (5830.960,1872.794)
        (5912.836,1894.363)
        (5996.014,1914.842)
        (6079.691,1932.421)
        (6163.066,1945.286)
        (6245.335,1951.625)
        (6325.695,1949.626)
        (6403.345,1937.476)
        (6477.481,1913.363)
        (6612.000,1822.000)

\drawline(6612,1822)    (6635.201,1686.299)
        (6612.000,1597.000)

\drawline(6612,1597)    (6588.799,1686.299)
        (6612.000,1822.000)

\drawline(6612,1822)    (6746.688,1914.001)
        (6821.008,1938.808)
        (6898.884,1951.812)
        (6979.495,1954.757)
        (7062.020,1949.386)
        (7145.639,1937.444)
        (7229.531,1920.674)
        (7312.875,1900.820)
        (7394.851,1879.627)
        (7474.637,1858.839)
        (7551.413,1840.198)
        (7692.651,1816.339)
        (7812.000,1822.000)

\drawline(7812,1822)    (7909.266,1853.001)
        (8032.312,1908.647)
        (8107.898,1947.910)
        (8195.203,1995.970)
        (8295.984,2053.707)
        (8412.000,2122.000)

\drawline(8712,7222)
        (8782.312,7076.688)
        (8843.250,6940.750)
        (8894.812,6814.188)
        (8937.000,6697.000)
        (8969.812,6589.188)
        (8993.250,6490.750)
        (9007.312,6401.688)
        (9012.000,6322.000)
        (9021.375,6190.750)
        (9049.500,6097.000)
        (9162.000,6022.000)
        (9162.000,6022.000)
        (9049.500,5947.000)
        (9021.375,5853.250)
        (9012.000,5722.000)
        (9008.484,5642.312)
        (8997.938,5553.250)
        (8980.359,5454.812)
        (8955.750,5347.000)
        (8924.109,5229.812)
        (8885.438,5103.250)
        (8839.734,4967.312)
        (8787.000,4822.000)

%\put(4962,4372){\makebox(0,0)[lb]{\smash{{{\SetFigFont{14}{16.8}{rm}(x,y)}}}}}
\put(9462,5872){\makebox(0,0)[lb]{\smash{{{\emph{h}}}}}}
\put(6462,1297){\makebox(0,0)[lb]{\smash{{{\emph{w}}}}}}
\put(4962,4372){\makebox(0,0)[lb]{\smash{{{\emph{(x,y)}}}}}}
\put(2112,3697){\makebox(0,0)[lb]{\smash{{{
\emph{(delta$_x$, delta$_y$)}}}}}}
\end{picture}
}

Example for edgeanchor 
\end{center}
% End of picture 

%\epsfxsize=200pt \epsfbox{nei_example.eps}

What are the values for 
\emph{delta$_x$} and 
\emph{delta$_y$} in this example?

The edgeanchor is left of the center of the node. The distance is 5/6 of
the nodewidth. Thus 
\emph{delta$_x$} is -5/6.

The edgeanchor is 2/4 below the center of the node. Since \Graphlet{} 
(and also Tcl/Tk) starts counting coordinates in the upper left corner 
we have 
\emph{delta$_y$} = 0.5.

Whenever the edge has to be redrawn (e.g. the node has moved) the edge is
clipped at the border of the node.

% --------------------------------------------------------------------------
% --------------------------------------------------------------------------

\section{Example for Changing the Edgeanchor}
\label{example_NEI}

Here is a first example. For the meaning of the functions have a look at 
the next sections.

The NEI is implemented in the classes 
\GT{Edge\_NEI} and \GT{Node\_NEI}. It is
accessed like all attributes of a node or edge.
 
Here is a small programm to set all edgeanchors to the center of the
node: (0.0, 0.0). 
In addition, the default function both for nodes and edges is set to leave 
the edgeanchors unchanged.

\begin{example}{e:SampleClassDeclaration}%
{A sample programm for the use of the NEI}
\begin{verbatim}
#include <gt_base/NEI.h>

int reset_NEIs (GT_Graph &g)
{
    GT_Node_NEI   *node_nei;
    GT_Edge_NEI   *edge_nei;
    edge          e;
    node          n;
    graph         *leda_graph = g.attached();

    forall_edges (e, *leda_graph) {
        edge_nei = g.gt(e).edge_nei();
        edge_nei->set_EA (GT_Source, 0.0, 0.0);
        edge_nei->set_EA (GT_Target, 0.0, 0.0);
        edge_nei->set_EA_default_function (
            GT_Keys::empty_function, GT_Source);
        edge_nei->set_EA_default_function (
            GT_Keys::empty_function, GT_Target);
    }

    forall_nodes (n, *leda_graph) {
        node_nei = g.gt(n).node_nei();
        node_nei->set_EA_default_function (
            GT_Keys::empty_function);
    }

    return GT_OK;
}

\end{verbatim}
\end{example}

\textbf{Note:} This function is already implemented.
\CSourceCodeLocation{}{base}{NEI}
\CIncludeStatement{}{base}{NEI}

%This function is already implemented.
%It is defined in 

%$<$gt\_base/NEI.h$>$.

% --------------------------------------------------------------------------
% Functions to change the edgeanchors
% --------------------------------------------------------------------------
\section{Functions to Change the Edgeanchor Once}

We devide the functions to change the edgeanchor into two subsets: 
One set from the point of view of an edge, 
the other set from the point of view of a node.

\textbf{Note:} 
This functions are used to change the edgeanchor once. 
The edgeanchor won't be changed if the edge is redrawn.
% ----------------------------------------
% ----------------------------------------
\subsection{Functions related to an edge}

There are three functions to change the edgeanchor:

\begin{Cdefinition}

  \item[int set\_EA (int \Param{where}, 
    double \Param{delta$_x$}, double \Param{delta$_y$})] \strut \\
    \emph{where} is either \emph{GT\_Source} or \emph{GT\_Target}. 
    It is used to determine whether the anchor is changed at the 
    source or target of the edge.
    The edgeanchor will be set to the relative point 
    (\emph{delta$_x$}, \emph{delta$_y$}).

    The return value is \emph{GT\_ERROR} if \emph{delta$_x$} or 
    \emph{delta$_y$} is not between -1 and 1.
    Otherwise the return value is \emph{GT\_OK}.

  \item[int set\_EA (int \Param{where}, GT\_Key \Param{key})] \strut \\
    \emph{where} is used the same way as in the function above. 
    This function sets the edgeanchor to a special point which is given by 
    \emph{key}. The possible values for \emph{key} are mainly given by the 
    compass:

    \textbf{Note:} Tcl/Tk switches north and south.
    Thus north is at the bottom of the node and south is at the top of it.
  
    \begin{ttdescription}
    \item[GT\_Keys::anchor\_center] \strut
      Place to the center of the node.
    \item[GT\_Keys::anchor\_n] \strut
      Place to the bottom of the node.
    \item[GT\_Keys::anchor\_ne] \strut
      Place to the lower right of the node.
    \item[GT\_Keys::anchor\_e] \strut
      Place to the right of the node.
    \item[GT\_Keys::anchor\_se] \strut
      Place to the upper right of the node.
    \item[GT\_Keys::anchor\_s] \strut
      Place to the top of the node.
    \item[GT\_Keys::anchor\_sw] \strut
      Place to the upper left of the node.
    \item[GT\_Keys::anchor\_w] \strut
     Place to the left of the node.
    \item[GT\_Keys::anchor\_nw] \strut
      Place to the lower left of the node.
    \end{ttdescription}

    The return value is \emph{GT\_ERROR} if \emph{key} is none of the 
    possibilities mentioned above. Otherwise the return value is \emph{GT\_OK}.

  \item[int set\_EA (GT\_Key \Param{source\_key}, 
    GT\_Key \Param{target\_key})] \strut\\
    This function allows to specify a rule for the computation of the values
    of the edgeanchor. \emph{source\_key} and \emph{target\_key} can be

    \begin{ttdescription}
    \item[EA\_next\_corner] \strut 
      Connect the last bend to the nearest corner of the node.
    \item[EA\_next\_middle] \strut 
      Connect the last bend to the closest middle of a side of the node.
    \item[EA\_orthogonal] \strut
      Connect the last bend by an orthogonal line to the node.
    \item[empty\_function] \strut
      Leave the edgeanchor unchanged.
    \end{ttdescription}
  
    \Graphlet{} then recomputes the edgeanchors with this rules for source
    and target. If it should connect an edge orthogonal and this is not 
    possible, the edge is connected to the next corner.

    The return value is \emph{GT\_ERROR} if \emph{source\_key} or 
    \emph{target\_key} is none of the keys mentioned above.
    Otherwise the return value is \emph{GT\_OK}.

\end{Cdefinition}

% ----------------------------------------
% ----------------------------------------
\subsection{Functions related to a node}

Up to now we only have the same functions for a node as we have for an edge.
The arguments of the functions are the same as for the functions related 
to an edge, except that we don't need the parameter \emph{where}. 
This is because each function is applied to this end of the edge which is
adjacent to the node. 
Only the last function influences source and target of the edge.

Here is a list of all functions:

\begin{Cdefinition}

  \item[int alledges\_set\_EA (double \Param{delta$_x$}, 
    double \Param{delta$_y$})] \strut
  \item[int alledges\_set\_EA (GT\_Key \Param{direction})] \strut
  \item[int alledges\_set\_EA (GT\_Key \Param{source\_key}, 
    GT\_Key \Param{target\_key})] \strut
\end{Cdefinition}  

% --------------------------------------------------------------------------
% The default function
% --------------------------------------------------------------------------

\section{Changing the Default Function}

The default function is a function which can change the edgeanchor every time
an edge has to be redrawn.
You can use this for example, if you want to connect two nodes by an edge
between the closest corners of the nodes.
\Graphlet{} computes the fitting corners each time the edge has to be redrawn.

You change the function with
\begin{Cdefinition}                
  \item[int set\_EA\_default\_function (GT\_Key \Param{function}, 
    int \Param{where})] \strut\\
    Again \emph{where} is GT\_Source or GT\_Target.
    \emph{function} can be one of the following:
    
    \begin{ttdescription}
      \item[GT\_Keys::empty\_function] \strut
      \item[GT\_Keys::EA\_next\_corner] \strut
      \item[GT\_Keys::EA\_next\_middle] \strut
      \item[GT\_Keys::EA\_orthogonal] \strut
    \end{ttdescription}
\end{Cdefinition}  

% --------------------------------------------------------------------------
% --------------------------------------------------------------------------
\section{Additional functions}

Here are some additional functions. 
They can be used to get some information on the NEI.

\CSourceCodeLocation{}{base}{edge\_NEI}
\CIncludeStatement{}{base}{NEI}

\begin{Cdefinition}

  \item[point get\_clip\_point (int \Param{where})] \strut\\
    For the given nodeposition and the given edgeanchor: Get the 
    coordinates where the edge is clipped.

  \item[point get\_EA (int \Param{where})] \strut\\
    Get the current edgeanchor either for source or target. 

  \item[double get\_EA\_x (int \Param{where})] \strut\\
    Get \emph{delta$_x$} of the edgeanchor either for source or target.

  \item[double get\_EA\_y (int \Param{where})] \strut\\
    Get \emph{delta$_y$} of the edgeanchor either for source or target.

  \item[GT\_Key get\_EA\_default\_function (int \Param{where})] \strut\\
    Get the key of the actual default function for source or target.

\end{Cdefinition}

Additionally we have a function to get the default function for the class
GT\_Node\_NEI:
\CSourceCodeLocation{}{base}{node\_NEI}
\CIncludeStatement{}{base}{NEI}

\begin{Cdefinition}
  \item[GT\_Key get\_EA\_default\_function ()] \strut\\
    Get the Key of the actual default function.
\end{Cdefinition}

% --------------------------------------------------------------------------
% --------------------------------------------------------------------------
\section{The Representation of Edgeanchors in gml-Files}

The NEI for nodes only consists of a default function. Thus you get an 
entry like the following for the nodes in the gml-file:

\begin{verbatim}
edgeAnchor [
defaultFunction "None"
]
\end{verbatim}

The NEI for edges consists of a default function for source and target.
Additionally we store (\emph{delta$_x$}, \emph{delta$_y$}) for source and 
target:

\begin{verbatim}
edgeAnchor [
sourceFunction "None"
targetFunction "Orthogonal"
xSource 0.800000
ySource -0.500000
xTarget 1.00000
yTarget -1.00000
]
\end{verbatim}

% --------------------------------------------------------------------------
% --------------------------------------------------------------------------
\section{Adjusting an Algorithm to the New NEI}

You have to do two things to adjust an algorithm to the new NEI:

\begin{enumerate}
  \item Make sure, that the default functions for nodes and edges which you 
    do not want to use are set to \emph{None} 
    (\emph{GT\_Keys::empty\_function}).

  \item Set the edgeanchor to the wanted point.

\end{enumerate}

Example:

You want to connect the nodes from node center to node center. The default
functions for nodes and edges are not allowed to change the edgeanchors.

In this case, use the  predefined function \emph{reset\_NEIs}. 
\CIncludeStatement{}{base}{NEI}
\emph{reset\_NEIs} is an implementation of the example in 
section \ref{example_NEI}.

\section{Changing the NEI with \GraphScript{}}

We have some additional attributes for nodes and edges with which you can 
access the NEI.

For further information on how to program in \GraphScript{} have a look
at the corresponding manual.

\subsection{Attributes for an edge}

\begin{TclAttributes}

  \Attribute{-delta\_x\_source}{number} \strut\\
  Real value in the interval -1.0, 1.0 .

  \Attribute{-delta\_x\_target}{number} \strut\\
  Real value in the interval -1.0, 1.0 .

  \Attribute{-delta\_y\_source}{number} \strut\\
  Real value in the interval -1.0, 1.0 

  \Attribute{-delta\_y\_target}{number} \strut\\
  Real value in the interval -1.0, 1.0 

  \Attribute{-source\_function}{TK\_Function} \strut\\
  Describes the function which is applied each time an edge is redrawn. 
    \begin{ttdescription}
    \item[EA\_next\_corner] \strut 
      Connect the last bend to the nearest corner of the node.
    \item[EA\_next\_middle] \strut 
      Connect the last bend to the closest middle of a side of the node.
    \item[EA\_orthogonal] \strut
      Connect the last bend by an orthogonal line to the node.
    \item[empty\_function] \strut
      Leave the edgeanchor unchanged.
    \end{ttdescription}
  
  \Attribute{-target\_function}{TK\_Function} \strut\\
  Describes the function which is applied each time an edge is redrawn. 
  \begin{ttdescription}
    \item[EA\_next\_corner] \strut 
    Connect the last bend to the nearest corner of the node.
    \item[EA\_next\_middle] \strut 
      Connect the last bend to the closest middle of a side of the node.
    \item[EA\_orthogonal] \strut
      Connect the last bend by an orthogonal line to the node.
    \item[empty\_function] \strut
      Leave the edgeanchor unchanged.
    \end{ttdescription}
 
\end{TclAttributes}

\subsection{Attributes for a node}

\begin{TclAttributes}

  \Attribute{default\_function}{TK\_Function} \strut\\
  Describes the function which is applied each time an edge, which is
  adjecent to the node, is redrawn. 
  \begin{ttdescription}
    \item[EA\_next\_corner] \strut 
    Connect the last bend to the nearest corner of the node.
    \item[EA\_next\_middle] \strut 
      Connect the last bend to the closest middle of a side of the node.
    \item[EA\_orthogonal] \strut
      Connect the last bend by an orthogonal line to the node.
    \item[empty\_function] \strut
      Leave the edgeanchor unchanged.
    \end{ttdescription}

\end{TclAttributes}

\end{document}










