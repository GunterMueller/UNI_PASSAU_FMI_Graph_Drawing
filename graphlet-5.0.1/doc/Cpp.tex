\documentclass[twoside,fleqn]{report}
\usepackage[report]{Graphlet}

\begin{document}

%%%%%%%%%%%%%%%%%%%%%%%%%%%%%%%%%%%%%%%%%%
%
% Programming Graphlet in C++
%
%%%%%%%%%%%%%%%%%%%%%%%%%%%%%%%%%%%%%%%%%%

% This file is generated automatically
\newcommand{\GraphletVersion}{1}



\title{Graphlet C++ Programmer Manual
  \\*[3.0cm]
  {\emph{DRAFT VERSION}}
  }

\author{Michael Himsolt\thanks{
    Graphlet is partially supported by the Deutsche
    Forschungsgemeinschaft, Grant Br 835/6-2,
    research cluster ``Efficient Algorithms for
    Discrete Problems and Their Applications''
    }
  }

\maketitle

\tableofcontents

%%%%%%%%%%%%%%%%%%%%%%%%%%%%%%%%%%%%%%%%%%
%
% Four Steps to Graphlet Applications
%
%%%%%%%%%%%%%%%%%%%%%%%%%%%%%%%%%%%%%%%%%%

\chapter{Introduction}
\label{c:Introduction}

This section describes how to implement the C++ part of
\Graphlet{} \emph{applications}.  A Graphlet application differs
from an \emph{applet} as follows. An applet is implemented solely
in \GraphScript{}, whereas applications are implemented in C++
and \GraphScript{}.  This section of the manual describes
\Graphlet{}'s C++ interface and shows how to implement new
\GraphScript{} commands in C++.

\begin{skills}
  \begin{description}

    \item[C++]
    Basics, Classes, Inheritance, Virtual Methods, Templates

    \item[LEDA]
    Basic data structures, Graphs

    \item[Tcl interface to C]
    Basics, Tcl interpreter construction

  \end{description}
\end{skills}


%
% Four Steps
%

\section{How to implement a \Graphlet{} Application}

A \Graphlet{} application typically consists of (1) a set of
algorithms, (2) \GraphScript{} commands for those algorithms and
(3) a user interface written in \GraphScript{}. Implementing a
GraphScript application consists of he

\begin{enumerate}
  
  \item \textbf{Implement algorithms in C++.}

  \begin{itemize}

    \item
    Algorithms are either implemented in pure LEDA or in a mixture of LEDA 
    and the \Graphlet{} C++ toolbox.
    
    \item As a rule of thumb, \Graphlet{}'s structures are needed
    if the graphical appearance of nodes, edges or and labels is
    relevant, or \Graphlet{}'s tool box is used.  This applies to
    all \emph{layout algorithms}, and graph theory algorithms
    which support \emph{algorithm animation}.

  \end{itemize}

  This step is described in Chapters \ref{c:Toolbox}, 
  \ref{c:GT_Graph}, \ref{c:Attributes} and 
  \ref{c:Algorithms}.


  \item \textbf{Provide \GraphScript{} commands for these
      algorithms.}

  \begin{itemize}
    
    \item Generally, there should a \GraphScript{} command for
    each algorithm. A family of algorithms may be converged into
    a single command.

    \item As a rule of thumb, the top level of a complex
    algorithm should always be implemented in \GraphScript{}.
    This allows easier customization than a C++ implementation,
    and helps to add a user interface.

  \end{itemize}

  This step is described in \ref{c:TclInterface}.
  
  
  \item \textbf{Build an extended \GraphScript{} interpreter
      which contains the new commands.}

  \begin{itemize}
    \item To do this, write a \texttt{main} routine, and link
    your code with \Graphlet{}, LEDA and Tcl/Tk.
  \end{itemize}
  
  This step is outlined in Chapters \ref{c:Modules} and
  \ref{c:Makefiles}.
  
  \item \textbf{Add a user interface with \GraphScript{}}

  \begin{itemize}
  
    \item Each algorithm should at least provide a window for the
    parameter settings. This window should be implemented with
    Tcl/Tk/\GraphScript{}.

  \end{itemize}

  This step is described in the GraphScript manual.
  
\end{enumerate}


%%%%%%%%%%%%%%%%%%%%%%%%%%%%%%%%%%%%%%%%%%
%
% \Graphlet{}'s C++ Toolbox
%
%%%%%%%%%%%%%%%%%%%%%%%%%%%%%%%%%%%%%%%%%%

\chapter{\Graphlet{}'s C++ Toolbox}
\label{c:Toolbox}



%
% Macros for class declaration
%

\section{Global Definitions}
\label{s:GlobalDefinitions}

\CSourceCode{the definitions presented in this section}{base}{Graphlet}


%
%
%

\subsection{Class Declaration Examples}

\begin{example}{e:SampleClassDeclaration}%
{A sample class declaration using Graphlet standards}
\begin{verbatim}
class GT_Sample_Class : public GT_Sample_Base_Class {

    GT_CLASS (GT_Sample_Class, GT_Sample_Base_Class);

    GT_VARIABLE (int, sample);
    GT_COMPLEX_VARIABLE (GT_GRAPH&, a_graph);

public:

    GT_Sample_Class ();
    virtual ~GT_Sample_Class ();

    void sample_method ();
}

\end{verbatim}
\end{example}


\begin{notes}

  \item All global declarations within Graphlet, especially class
  names, \textbf{must} use the prefix \texttt{GT\_}. Other
  projects are exempt from this treaty, but we strongly recommend
  to use a common prefix as well.
  
  \item Graphlet's coding standards require that class names
  (with \texttt{GT\_} stripped) start with a capital letter.
  
  \item Always provide a constructor and a desctructor. The
  latter one must be \texttt{virtual} if virtual methods are
  used within the class.

\end{notes}


%
% Class Declarations
%

\subsection{Macros for class declaration}

\begin{Cdefinition}

  \item[\GT{BASE\_CLASS (\Param{class})}] \strut\\
  The macro \GT{BASE\_CLASS (class)} implements standard 
  declarations for a class which is not derived from another \Graphlet{} 
  class.
  
  \item[\GT{CLASS (\Param{class}, \Param{base})}] \strut\\
  The macro \GT{CLASS (\Param{class}, \Param{base})} implements
  standard declarations for a class which is derived from another
  \Graphlet{} class \emph{base}.  \GT{CLASS} implements the
  following:

  \begin{ttdescription}

    \item[baseclass] \strut\\
    This is a local \texttt{typedef} for the base class.  For
    example,
    \begin{quote}
      \texttt{baseclass::}\emph{method}      
    \end{quote}
    refers to \emph{method} in the class \emph{base}. For
    example, a virtual function \emph{f} which extend the
    functionality of the base class method will usually call
    \begin{quote}
      \texttt{baseclass::}\emph{f}\texttt{($\ldots$);}
    \end{quote}
    at the beginning or at the end of the method.
  \end{ttdescription}
  
\end{Cdefinition}
  

\begin{notes}
  \item Graphlet does not encourage to use multiple inheritance
  widely, so there is no support for this type of class
  declaration.
\end{notes}

%
% Member Variables
%

\subsection{Member Variables}
\label{s:MemberVariables}

\Graphlet{} defines the macros \GT{VARIABLE} shortcuts to declare 
member variables and corresponding accessor methods:
  
\begin{Cdefinition}

  \item[\GT{VARIABLE (\Param{type}, \Param{name})}] \strut\\
  This macro defines the following:
  
  \begin{ttdescription}
    \item[type the\_\Param{name};] \strut\\
    This is the private member variable.
    
    \item[type \Param{name}() const] \strut\\
    This is the public accessor method for \texttt{name}.

    \item[virtual void \Param{name} (\Param{type})] \strut\\
    This is the public method which is used to set \texttt{name}.
  \end{ttdescription}
  
  \item[\GT{COMPLEX\_VARIABLE (\Param{type}, \Param{name})}] \strut\\
  This macro defines the following:
  
  \begin{ttdescription}
    \item[type the\_\Param{name};] \strut\\
    This is the private member variable.
    
    \item[const \Param{type}\& \Param{name}() const] \strut\\
    This is the public accessor method for \texttt{name}.

    \item[virtual void \Param{name} (const \Param{type}\&)] \strut\\
    This is the public method which is used to set \texttt{name}.
  \end{ttdescription}
  
\end{Cdefinition}

\begin{notes}

  \item The prefix \texttt{the\_} is required by \Graphlet{}'s naming 
  conventions.
  
  \item \GT{COMPLEX\_VARIABLE} is identical to \GT{VARIABLE} with
  the exception that is uses \texttt{const\&} semantics in the
  accessor methods.  This is done to avoid extensive copying of
  non-simple classes.  Generally, \GT{VARIABLE} should only be
  used for simple data types such as \texttt{int},
  \texttt{double} and pointers, and \GT{COMPLEX\_VARIABLE} for
  all others.
  
  \item By default, \Graphlet{} does not export
  non-\texttt{const} references to member variables.  One reason
  for this is that \Graphlet{} often needs to react to a change,
  and a change can only be recognized if it is done through a
  dedicated method.
  
\end{notes}



%
% GT_Status
%

\subsection{The class \GT{Status}}

\begin{Cdeclaration}{Status}{enumeration \GT{Status}}
\begin{verbatim}
enum GT_Status {
    GT_OK    = 0,
    GT_ERROR = 1
};
\end{verbatim}
\end{Cdeclaration}

\noindent The enumeration \GT{Status} is modeled after the Tcl return
values \texttt{TCL\_OK} and \texttt{TCL\_ERROR}.  \GT{Status}
provides status return values which is compatible to those used
in Tcl, and should be used whenever a return value must be
converted into a Tcl return value at a later point.

\begin{notes}

  \item Use \GT{Status} only when compatibility with Tcl is necessary.  
  In all other cases, return a boolean value or a more descriptive 
  enumeration.
  
  \item Graphlet reserves the right to add more values
  \GT{Status}. Application programs should \emph{not} no that.
  
  \item \GT{Status} is \textbf{not} compatible with the builtin
  C++ data type \texttt{bool}.

  \item The main reason for using \GT{Status} instead of
  \texttt{TCL\_OK} and \texttt{TCL\_ERROR} is to prevent Tcl from
  being included in the \Graphlet{} base libraries, and makes
  \Graphlet{} less dependent on Tcl.

\end{notes}


%
% The class GT
%

\section{The Class \texttt{GT} and the Global Variable \texttt{graphlet}}

Graphlet provides a global class \texttt{GT} and a global
variable\footnote{Of course the Coding Standards (Chapter
  \ref{c:CS:C++}) forbid global variables. Well, nobody is
  perfect. Also remember: ``\textsc{Quod licet jovi, non licet
    bovi}''} \texttt{graphlet} which contain utility methods and
hold some global data.  Declaration \ref{Cdeclaration:GT} shows
the declaration of class \texttt{GT}.

\begin{Cdeclaration}{GT}{class \texttt{GT}}
\begin{verbatim}
class GT
{
    GT_BASE_CLASS (GT);
public:
    static char* strsave (const char* s, int max_length = 0);
    static bool streq (const char* s1, const char* s1);

    GT_Id id;
    GT_Error error;
    GT_Keymapper keymapper;
    GT_GML* gml;
    GT_Parser* parser;
};

GT* graphlet
\end{verbatim}
\end{Cdeclaration}


\begin{Cdefinition}

  \item[char* strsave (const char* s, int max\_length=0)] \strut\\
  \texttt{strsave} is a wrapper for copying C strings.
  \texttt{strasave} copies at most \texttt{max\_length}
  characters of the string \texttt{s} into a new string which is
  allocated using \texttt{malloc}. If \texttt{max\_length} is
  omitted or is \texttt{0}, the whole string is copied. In both
  cases, a trailing \verb|'\0'| character is added.

  \item[bool streq (const char* s1, const char* s1)] \strut\\
  Wrapper for \texttt{!strcmp(s1,s2)}.

  \item[graphlet->id]  \strut\\
  The object \texttt{graphlet->id} manages unique \texttt{id} numbers.
  Each time the method \texttt{graphlet->id.next\_id()} is called, a
  new number is emitted.  Graphlet uses \texttt{id} to assign
  unique identifiers to graphs, nodes, edges and user interface
  objects.
  
  \item[graphlet->error]  \strut\\
  This object holds a dictionary of predefined error messages.
  See also section \ref{s:Error}.

  \item[graphlet->keymapper] \strut\\
  The keymapper is a hash table of often used strings. See also
  Sections \ref{s:Keymapper} and \ref{s:PredefinedKeys}.

  \item[graphlet->parser] \strut\\
  This is the GML parser. \NYD.
  
  \item[graphlet->gml] \strut\\
  This object contains additional information for procedures
  which output graphs in the GML file format.  \NYD.

\end{Cdefinition}



%
% The Keymapper
%

\section{The Keymapper}
\label{s:Keymapper}

\CSourceCodeLocation{}{base}{Keymapper}
\CIncludeStatement{}{base}{Graphlet}

\noindent \Graphlet{} uses a hash table to store often used strings.
This hash table is called the \emph{keymapper}, since it was
first implemented to store the keys in the the GML file
format. A \emph{key} is an element of this hash table.
Mathematically, a keymapper is defined as follows:

\[
  \mbox{key} \underbrace{\longmapsto}_{\mbox{keymapper}} \mbox{string}
\]


\noindent \Graphlet{} uses keys for the following purposes:

\begin{description}
  
  \item[Efficiency] Keys require fewer storage than strings and are 
  much more efficient to compare.
  
  \item[Flexible enumerations] C++ enumerations are limited in
  that they cannot be extended (as opposed to classes).  Keys can
  be used effectively in place of enumerations (with the
  restrictions that they may not used in \texttt{case}
  statements, and canot be conterted to integers).
  
  \item[Precomputed Information] \Graphlet{} evaluates a key at the 
  time it is stored in the repository, and can store additional 
  information on the key.  For example, a GML key which starts with 
  a lowercase letter is \emph{safe}, white one that starts with an 
  upper case letter is not.

  \item[String values] Keys can be converted to LEDA strings.
  Especially, they can be written to files.

\end{description}

\begin{notes}
  \item Due to their nature, keys are not type safe.  The C++
  compiler cannot detect a wrong key, as it could do with an
  enumeration.  However, this does not necessarily mean that keys
  are unsafe by all means; correctness can be enforced with
  proper \texttt{assert} statements.
\end{notes}

%
% GT_Keymapper
%

\subsection{The class \GT{Keymapper}}

Keys stored in a global object of type \GT{Keymapper}.  The
following methods are available for the class \GT{Keymapper}:

\begin{Cdefinition}

  \item[\GT{Key} add (const string\& \Param{s})] \strut \\
  The method \texttt{add} adds a new key with the name \emph{s}
  to a keymapper, and returns the key. If a key with the same
  name already exists, it returns the existing key.

\end{Cdefinition}

\begin{notes}
  \item There is no way to remove a key from a keymapper. This is
  because the integrity of Graphlet's data structures can only be
  guaranteed if a key keeps its value until the end of the
  program.
  
  \item Dont use keys for temporary objecte like labels; this
  would make the keymapper unneccessarily large.
  
  \item There should be only a single global object of type
  \GT{Keymapper}.
\end{notes}




%
% How to add a new Key
%

\subsection{How to add a new Key}
\label{s:HowToAddANewKey}

To add a new key, use the followin C++ code:

\begin{verbatim}
#include <gt_base/Graphlet.h>

GT_Key new_key = graphlet->keymapper.add (
    "This is the name of the key");
\end{verbatim}


%
% The class GT_Key
%

\subsection{The class \GT{Key}}

The following methods are available for the class \GT{Key}:

\begin{Cdefinition}
  
  

  \item[\GT{Key()}] \strut\\
  This constructor creates a new key which is not yet
  \emph{defined}, i.e. it has no name. See section
  \ref{s:HowToAddANewKey} how to create a key with a name.
  

  \item[bool defined() const] \strut\\
  The method \texttt{defined} tests wether the key has been
  assigned a name. Generally, keys created with the constructor
  \GT{Key()} will return \texttt{false}, while keys created with
  from \GT{Keymapper::add} will return \texttt{true}. See also
  \texttt{active} below.

  \item[bool active() const] \strut\\
  The method \texttt{active} tests wether a key has been assigned
  a name (it is \texttt{defined}) \textbf{and} is not the key
  \GT{Keys::def}
  
  \item[const string\& name() const] \strut\\
  The method \texttt{named} returns the name of the key. This
  method may only be exceuted if \texttt{defined} is true.
  

  \item[bool operator== (const \GT{Key}\& other\_key) const] \strut\\
  The operator \texttt{==} compares keys. This is faster than to
  compare the names of the keys. Two keys are equal if (a) they
  have been created with the same keymapper and (b) their names
  are equal.
  

  \item[bool operator!= (const \GT{Key}\& other\_key) const] \strut\\
  The operator \texttt{!=} compares keys. This is faster than to
  compare the names of the keys. Two keys are not equal if (a)
  they have been created with different keymappers or (b) their
  names are nor equal.
  

  \item[const \GT{Key\_class}* description () const] \strut\\
  Returns a pointer to the \texttt{description} of the key (see
  also section \ref{s:GT_Key_description}).

\end{Cdefinition}

\begin{notes}
  
  \item New keys should always be created with the \texttt{add}
  method of keymapper. The constructor \verb|GT_Key()| may be
  used to create an undefined key.
  
  \item There is no way to make an undefined key defined; assign a
  defined key instead, as in
\begin{verbatim}
GT_Key sample_key;
sample_key = graphlet->keymapper.add ("sample_key");
\end{verbatim}
  
  \item There is no restriction on the name of a key; names which
  are used as GML keys must however consist of up to 127 letters
  and digits only, and the first character must be a letter.
\end{notes}


\subsection{The class \GT{Key\_description}}
\label{s:GT_Key_description}

The only member variable in the class \GT{Key} is a pointer to the 
description of the key. This description is an object of type
\GT{Key\_description}, and can be accessed through the method
\GT{Key::description}. The class \GT{Key\_description} provides the 
following features:

\begin{Cdefinition}

  \item[const string\& name () const] \strut\\
  Returns the \emph{name} of the key; the method \GT{Key::name}
  is usually a more convenient way to access the name of a key.
  

  \item[bool save() const] \strut\\
  Returns wether the key is save in the sense of GML. Roughly
  speaking, as safe key represents data which will remain
  consistent when other attributes or the graph topology change.
  
  Safe keys start with a lowercase letter, while unsafe keys
  start with a capital letter.  See the GML manual for details.
  There is no way to set this information, it is computed from
  the name of the key.
  
  \item[bool visible() const] \strut\\
  Returns wether the key is \emph{visible}. A key is visible if
  its name starts with a '\@' character and invsisible otherwise.
  Invisible keys are not written to GML files.

\end{Cdefinition}


%
% Predefined Keys
%

\section{Predefined Keys}
\label{s:PredefinedKeys}

\CSourceCodeLocation{}{base}{Keys}
\CIncludeStatement{}{base}{Graphlet}

\noindent \Graphlet{} defines a class \GT{Keys} which holds a large
number predefined keys.  Each key is defined as a static member
variable, and can be accessed as

\begin{quote}
  \GT{Keys::}\emph{key}
\end{quote}


%
% Very special keys
%

\subsubsection{Special keys}

The following special keys are used by Graphlet:

\begin{ttdescription}
\item[\GT{Key} \GT{Keys}::def] \strut\\
This key indictates that the \emph{default} value should be used.
\end{ttdescription}


%
% GML Tags
%

\subsubsection{GML Tags}

Generally, \Graphlet{} defines a key for each GML tag which is
used by \Graphlet{}.  Each tag has the form

\begin{quote}
  \GT{Key::}\emph{tag}
\end{quote}

\noindent where \emph{tag} is the name of the GML tag.
Graphlet also defines tags \texttt{option\_tag} for each tags;
these are the corresponding Tcl options (``-\texttt{tag'')}.
Especially, the following keys are defined in \GT{keys}:

\begin{verbatim}
static GT_Key graphics, label_graphics;
static GT_Key graph;
static GT_Key node;
static GT_Key edge;

static GT_Key version, option_version;
static GT_Key creator, option_creator;
static GT_Key id, option_id;
static GT_Key uid, option_uid;
static GT_Key label_uid, option_label_uid;
static GT_Key name, option_name;
static GT_Key label, option_label;
static GT_Key graph_attrs, option_graph_attrs;
static GT_Key node_attrs, option_node_attrs;
static GT_Key edge_attrs, option_edge_attrs;
static GT_Key center, option_center;

static GT_Key directed, option_directed

static GT_Key x, option_x;
static GT_Key y, option_y;
static GT_Key w, option_w;
static GT_Key h, option_h;
\end{verbatim}


%
% Keys for colors
%

\subsubsection{Colors}

\begin{alltt}
GT_Keys::white;
GT_Keys::black;
GT_Keys::red;
GT_Keys::green;
GT_Keys::blue;
\end{alltt}

%
% Graphic Objects
%

\subsubsection{Graphic objects}

\begin{alltt}
GT_Keys::type_arc;
GT_Keys::type_bitmap;
GT_Keys::type_image;
GT_Keys::type_line;
GT_Keys::type_oval;
GT_Keys::type_polygon;
GT_Keys::type_rectangle;
GT_Keys::type_text;
\end{alltt}

\noindent See also section \ref{s:Attributes:CommonGraphics}.

\subsubsection{Anchors}

\Graphlet{} predefines the following keys for anchors:

\begin{alltt}
// Node labels
GT_Keys::anchor_center;
GT_Keys::anchor_n;
GT_Keys::anchor_ne;
GT_Keys::anchor_e;
GT_Keys::anchor_se;
GT_Keys::anchor_s;
GT_Keys::anchor_sw;
GT_Keys::anchor_w;
GT_Keys::anchor_nw;
\end{alltt}

\begin{alltt}
// Edge labels
GT_Keys::anchor_first;
GT_Keys::anchor_last;
\end{alltt}

\begin{alltt}
// Edge anchors
GT_Keys::anchor_clip;
GT_Keys::anchor_corners;
GT_Keys::anchor_middle;
GT_Keys::anchor_8;
\end{alltt}



%
% GT_Point
%

\section{The class \GT{Point}}

\CSourceCode{class \GT{Point}}{base}{Geometry.h}

The class \GT{Point} implements 2 dimensional points with 
\texttt{double} coordinates.

\begin{Cdefinition}
\item[\GT{Point()}] \strut\\
  Creates a new point at $(0.0,0.0)$.
\item[\GT{Point} (double \Param{x}, double \Param{y})] \strut\\
  Creates a new point at $(x,y)$.
\item[\GT{Point}(const point\& \Param{p})] \strut\\
  Creates a new point from a LEDA \texttt{point} object.
\item[\GT{Point} (const vector\& \Param{v})] \strut\\
  Creates a new point from a LEDA \texttt{vector} object.
\item[void x (double \Param{x})] \strut\\
  Set the x coordinate of a point.
\item[void y (double \Param{y})] \strut\\
  Set the y coordinate of a point.
\item[double x() const] \strut\\
  Returns the \texttt{x} coordinate of a point.
\item[double y() const] \strut\\
  Returns the \texttt{y} coordinate of a point.

  \item[void move (const vector\& \Param{move\_xy})] \strut\\
  Move the point by a vector \emph{move\_xy}.

  \item[virtual void scale (double scale\_by, const point\& origin = point (0.0,0.0))] \strut\\
  Scale the coordinates of the point by a factor
  \texttt{scale\_by}, with respect to \texttt{origin}.

\end{Cdefinition}


\begin{notes}
  
  \item In the current implementation, \GT{Point} is derived from
  the LEDA class \texttt{point}. This might change in the future.

  \item \GT{Point} differs from LEDA's \texttt{point} class in that 
  LEDA does not provide operations which change the coordinates of a 
  point.

\end{notes}


%
% GT_Polyline
%

\section{The class \GT{Polyline}}

\CSourceCode{class \GT{Polyline}}{base}{Geometry}

The class \GT{Polyline} implements 2-dimensional polylines or 
polygons.  \GT{Polyline} is derived from \verb|list<GT_Point>| and 
provides the following methods:

\begin{Cdefinition}

  \item[GT\_Polyline ()] \strut\\
  Creates an empty polyline.
  \item[GT\_Polyline (const GT\_Polyline\& \Param{l})] \strut\\
  Copy constructor.
  \item[GT\_Polyline (const list<GT\_Point>\& \Param{l})] \strut\\
  Creates a polyline from a list of \GT{Point} objects.
  \item[virtual \~GT\_Polyline ()] \strut\\
  Destructor.

  \item[segment nth\_segment (const int \Param{n} const] \strut\\
  Returns the n\emph{th} segment of the line, that is the segment from 
  point $n$ to $n+1$.

  \item[void move (const vector\& \Param{move\_xy}] \strut\\
  Move the polyline by a vector \emph{move\_xy}.

  \item[virtual void scale (double scale\_by, const point\& origin = point (0.0,0.0))] \strut\\
  Scale the line by a factor \texttt{scale\_by}, with respect to
  \texttt{origin}.

\end{Cdefinition}

\begin{notes}
  \item A \GT{Polyline} object must contain at least 2 points.
\end{notes}

%
% GT_Rectangle
%

\section{The class \GT{Rectangle}}

\CSourceCode{class \GT{Rectangle}}{base}{Geometry}

The class \GT{Rectangle} implements 2-dimensional rectangles.  
\GT{Rectangle} is derived from \GT{Point} and provides the following 
methods:

\begin{Cdefinition}

  % Constructor / Destructor

  \item[GT\_Rectangle ()] \strut\\
  Creates an empty rectangle at position $(0,0)$.

  \item[GT\_Rectangle (const point\& \Param{p}, double \Param{w}, double \Param{h})] \strut\\
  Creates a rectangle with width \emph{w} and height \emph{h} at 
  position \emph{p}.

  \item[GT\_Rectangle (double \Param{x}, double \Param{y}, double \Param{w}, double \Param{h})] \strut\\
  Creates a rectangle with width \emph{w} and height \emph{h} at 
  position $(x,y)$.

  \item[virtual \~GT\_Rectangle ()] \strut\\
  Virtual destructor.

  % w, h

  \item[void w (double \Param{new\_width})] \strut\\
  Set the width of the rectangle.
  \item[double w () const] \strut\\
  Return the width of the rectangle.
  \item[void h (double \Param{new\_height})] \strut\\
  Set the height of the rectangle.
  \item[double h () const] \strut\\
  Return the height of the rectangle.

  % Move and scale

  \item[void move (const vector\& \Param{move\_xy}] \strut\\
  Move the rectangle by a vector \emph{move\_xy}.

  \item[virtual void scale (double scale\_by,
  const point\& origin = point (0.0,0.0))] \strut\\
  Scale the rectangle by a factor
  \texttt{scale\_by}, with respect to \texttt{origin}.

  % Utilities

  \item[bool includes (const point\& \Param{p}) const] \strut\\
  Returns \emph{true} if the point \emph{p} lies within the 
  rectangle, and \emph{false} otherwise.  

  \item[void expand (double x, double y)] \strut\\
  Expands the rectangle by adding \texttt{x} to the width and
  \texttt{y} to the height of the rectangle.

  \item[void union\_with (const GT\_Rectangle\& rect)] \strut\\
  Creates the union with \texttt{rect}.

  % anchor points

  \item[point anchor\_c() const;] \strut\\
  Returns the coordinates which correspond to a Tk \emph{c} anchor point.
  \item[point anchor\_n() const;] \strut\\
  Returns the coordinates which correspond to a Tk \emph{n} anchor point.
  \item[point anchor\_ne() const;] \strut\\
  Returns the coordinates which correspond to a Tk \emph{ne} anchor point.
  \item[point anchor\_e() const;] \strut\\
  Returns the coordinates which correspond to a Tk \emph{e} anchor point.
  \item[point anchor\_se() const;] \strut\\
  Returns the coordinates which correspond to a Tk \emph{se} anchor point.
  \item[point anchor\_s() const;] \strut\\
  Returns the coordinates which correspond to a Tk \emph{s} anchor point.
  \item[point anchor\_sw() const;] \strut\\
  Returns the coordinates which correspond to a Tk \emph{sw} anchor point.
  \item[point anchor\_w() const;] \strut\\
  Returns the coordinates which correspond to a Tk \emph{w} anchor point.
  \item[point anchor\_nw() const;] \strut\\
  Returns the coordinates which correspond to a Tk \emph{nw} anchor point.
  \item[double top() const] \strut\\
  Return the y coordinate of the top side (i.e.\ the minimum y coordinate) 
  of the rectangle.
  \item[double right() const] \strut\\
  Return the x coordinate of the right side (i.e.\ the maximum x coordinate) 
  of the rectangle.
  \item[double bottom() const] \strut\\
  Return the y coordinate of the bottom side (i.e.\ the maximum y 
  coordinate) of the rectangle.
  \item[double left() const] \strut\\
  Return the y coordinate of the bottom side (i.e.\ the minimum y 
  coordinate) of the rectangle.

\end{Cdefinition}

\begin{notes}
  \item The methods \texttt{top}, \texttt{right}, \texttt{bottom} and 
  \texttt{left} are not compatible with the \texttt{anchor\_n}, 
  \texttt{anchor\_e}, \texttt{anchor\_s} and \texttt{anchor\_r}, 
  respectively. Roughly speaking, they go in the opposite directions.
\end{notes}


%%%%%%%%%%%%%%%%%%%%%%%%%%%%%%%%%%%%%%%%%%
%
% Using Graphs
%
%%%%%%%%%%%%%%%%%%%%%%%%%%%%%%%%%%%%%%%%%%

\chapter{The class \GT{Graph}}
\label{c:GT_Graph}

\CSourceCode{}{base}{Graph}


%
% The class \GT{Graph}
%

\section{The Class \GT{Graph}}

LEDA's \texttt{graph} class is not really suitable for complex
interactive applications. For example, it lacks a mechanism to
store attributes in nodes and edges. Therefore, Graphlet provides
another data structure \GT{Graph} which extends LEDA graphs with
a rich set of attributes.

Unlike other graph classes within LEDA, \GT{Graph} does not
derive from \texttt{graph}, but is a separate structure which is
attached to a LEDA graph, that is it has a pointer to a LEDA
graph.  Whenever the graph topology in the LEDA graph changes,
the \GT{Graph} structure is notified and adjusts itself.  This
approach has several advantages:

\begin{itemize}
  
  \item It makes it possible to use graph classes which are
  derived from LEDA's \texttt{graph} class. This could also have
  been achieved by using templates, but not all current compilers
  handle large template structures well. Also, this would lead to
  excessive code duplication in the compiled program.

  \item The interface becomes much cleaner.  \GT{Graph} has only few 
  dependencies on LEDA's graph structure.

  \item Our strategy is backward compatible with existing LEDA 
  algorithms.  Graphlet \emph{can} easily use LEDA algorithms which 
  are not designed for Graphlet, although they will have no access to 
  the user interface.
  
  \item It becomes possible\footnote{This feature is not
    implemented at the moment.} to change the graph class at
  runtime.
  
  \item The data structure is not yet fully exploited.
  Technically possible\footnote{Not implemented in the current
    version.} extenion capabilities include schemes where one
  graph has several independend graphical representations, and
  several graph data structures share one graphical
  representation.

\end{itemize}

\GT{Graph} provides support for arbitrary graph, node and edge
attributes, and provides the device independent methods for
displaying graphs. There are two classes of attributes:
\Graphlet{} attributes and user defined attributes. See 
\ref{c:G++:Attributes} for details.

%
%
%

\section{Creating and Initializing GT\_Graph}
\label{s:CreatingAndInitializingGraph}

This section shows how to create and initialize a \GT{Graph} object.  
Programmers who write algorithms which manipulate a given graph may 
skip this section.  Especially, the class \verb|GT_Tcl_Interface<>| 
will provide initialize the graph by itself.  See Chapter 
\ref{c:TclInterface} for details.

The following steps are necessary to use a GT\_Graph with a LEDA graph 
class:

\begin{enumerate}
  
  \item
  \emph{Optional.} Declare a class \texttt{C} which is
  derived from LEDA's \texttt{graph} class.
  
  \item \label{l:GraphInit:Shuttle} Use the class \GT{Shuttle} to
  construct a graph class which can communicate with \GT{Graph}:

  \begin{quote}
    \verb|GT_Shuttle* c_shuttle = new GT_Leda_Shuttle<C>;|
  \end{quote}
  
  The class \GT{Leda\_Shuttle} class adds communication
  capabilities to the class \texttt{c}. Any class derived from
  \GT{Shuttle} can notify a \GT{Graph} when nodes and edges are
  deleted.
  
  \item \label{l:GraphInit:newGraph} Initialize the GT\_Graph as follows:
  \begin{quote}
    \verb|GT_Graph* gt_graph = new GT_Graph;|
  \end{quote}  
  This creates a new \GT{Graph} which is not yet attached to a
  LEDA graph class.

  \item \label{l:GraphInit:connect}
  Connect the \texttt{c\_shuttle} and the \gt{graph} graph:
  \begin{quote}
    \verb|gt_graph->leda (*c_shuttle);|
  \end{quote}
  This step establishes communication between the
  \GT{graph} and \texttt{c\_shuttle}.
    
  \item \label{l:GraphInit:initialize} Initialize the graph:
  \begin{quote}
    \verb|gt_graph->new_graph();|
  \end{quote}
  This step is neccessary because the method
  \texttt{GT\_Graph::new\_graph()} is virtual and this cannot be
  used from the constructor.
  
\end{enumerate}

\begin{notes}
  
  \item \label{n:InitializingGraph} It is technically possible to
  skip step \ref{l:GraphInit:Shuttle} and use a graph class which
  is not derived from \GT{Leda\_Shuttle} in step
  \ref{l:GraphInit:connect}. However, this may only be used for
  static graphs, since \GT{Graph} cannot recognize insertions and
  deletions. \emph{Use this only if you have to, and if you know
    what you are doing}.
  
  \item Any class derived from \GT{Graph} may be used in step
  \ref{l:GraphInit:newGraph}. To use the Tcl interface,
  initialize with \GTTcl{Graph} instead of \GT{Graph}.  See
  also Chapter \ref{c:TclInterface} for details.

\end{notes}


%
% Using GT_Graph
%

\section{How to use \GT{Graph}}
\label{s:UsingGTGRaph}

\subsection{How to access the LEDA graph structure}

The LEDA graph associated with a Graphlet graph can be accessed
through the methods \texttt{leda} and \texttt{attached}:

\begin{Cdefinition}

  \item[graph\& GT\_Graph::leda()] \strut\\
  This method returns a reference to the underlying LEDA graph.  It is 
  legal to make changes in the graph if graph has been initialized 
  with a \GT{Shuttle} structure, which is the default in \Graphlet{} 
  (see section \ref{s:CreatingAndInitializingGraph}, especially 
  note \ref{n:InitializingGraph}).
  
  \item[const graph\& GT\_Graph::leda() const] \strut\\
  This method returns a constant reference to the underlying LEDA
  graph. As aleays, this is the preferred way to access a graph.

  \item[void GT\_Graph::leda (graph* \Param{g})] \strut\\
  Connects the LEDA graph \emph{g} with a \GT{Graph} object. See
  section \ref{s:CreatingAndInitializingGraph} for details.
  If this method is used, then changes in the leda graph
  \emph{cannot} be recognized by the \GT{Graph} object.
  
  \item[void GT\_Graph::leda (graph\& \Param{g})] \strut\\
  Same as above, but uses  a reference instead of a pointer.
  
  \item[void GT\_Graph::leda (GT\_Shuttle\& \Param{g})] \strut\\
  Connects the LEDA graph \emph{g} with a \GT{Shuttle} object.
  See section \ref{s:CreatingAndInitializingGraph} for
  details. This method should always be preferred over the last
  two ones, since it enables \GT{Graph} to detect changes in the
  graph structure.
  
\end{Cdefinition}

\noindent Use the following shown in example
\ref{e:InsertingAndDeletingNodesAndEdge} to insert or delete
nodes and edges in a \GT{Graph}. The general approach is as
follows.

\begin{enumerate}
  \item Get a \emph{reference} to the LEDA graph via the method
  \GT{graph::leda()}. It is important to get a reference; the
  statement \verb|graph g = gt_graph.leda();| will construct a
  \textbf{copy} of the graph.

  \item  Perform the changes in this graph.
\end{enumerate}

\begin{example}{e:InsertingAndDeletingNodesAndEdges}%
{A simple example showing how to insert nodes and edges}
\begin{verbatim}
void sample (GT_Graph& gt_graph)
{
    graph& g = gt_graph.leda();
    
    node n;
    forall_nodes (n, g) {
        g.delete_node (n);
    }
    
    for (int i=0; i<42; i++) {
        g.new_node ();
    }
}
\end{verbatim}
\end{example}

%
% Source and Target
%

\subsection{Source and Target nodes}

\Graphlet{} always makes a distinction between source and target 
nodes, even in undirected graphs.  This is done because when an edge 
is drawn, the coordinate assignment (\emph{``from (x1,y1) to 
(x2,y2)''}) imposes a direction, even on an undirected graph.


%
% Copy
%

\subsection{Copy Operations}

\GT{Graph} provides copy operations for nodes and edges. These
operations will also copy \emph{all} attributes which are
associated with the node or edge.

\begin{Cdefinition}
  

  \item[virtual node copy (node n, \GT{Graph}\& into\_graph);] \strut\\
  Creates a copy of \texttt{old\_node} in graph
  \texttt{into\_graph} and returns it.
  
  \item[virtual edge copy (edge e, node source, node target,
  \GT{Graph}\& into\_graph);] \strut\\
  Creates a copy of \texttt{old\_edge} in graph
  \texttt{into\_graph} and returns it.  The endnodes of the new edge are
  \texttt{source} and \texttt{target},

\end{Cdefinition}

\noindent In both methods, \texttt{into\_graph} may be \texttt{*this};
in this case, a copy operation \emph{within} the same graph is
accomplished.


%
% Utilities
%

\subsection{Graph Utilities}

\subsubsection{Bounding Box}

The bounding box is the smalles rectangle in which the graph is
enclosed. \GT{Graph} provides several methods to compute the
bounding box of graphs, nodes and edges:

\begin{Cdefinition}

  \item[GT\_Rectangle bbox () const;] \strut\\
  Computes the bounding box of the graph.

  \item[GT\_Rectangle bbox (node n) const;]\ \strut\\
  Computes the bounding box of node \texttt{n}.

  \item[GT\_Rectangle bbox (edge e) const;] \strut\\
  Computes the bounding box of edge \texttt{e}.

  \item[virtual void bbox (GT\_Rectangle\& bbox) const;] \strut\\
  Computes the bounding box of the graph, and returns it in \texttt{bbox}.  

  \item[virtual void bbox (node n, GT\_Rectangle\& bbox) const;] \strut\\
  Computes the bounding box of node \texttt{n}, and returns it in
  \texttt{bbox}.

  \item[virtual void bbox (edge e, GT\_Rectangle\& bbox) const;] \strut\\
  Computes the bounding box of edge \texttt{e}, and returns it in
  \texttt{bbox}.

\end{Cdefinition}


%
% Search
%

\subsubsection{Search for a node or edge with a given id.}
 
The following methods search for a node resp.\ edge with a given
id (see also chapter \ref{c:G++:Attributes} on attributes). They
return the corresponding object, if found, and \texttt{0}
otherwise.

\begin{Cdefinition}

  \item[node find\_node (const int id);] \strut\\
  Searches for a node \emph{n} with \verb|gt(|\emph{n}\verb|).id() == id|.

  \item[edge find\_edge (const int id);] \strut\\
  Searches for an edge \emph{e} with \verb|gt(|\emph{e}\verb|).id() == id|.

\end{Cdefinition}

\begin{notes}
  
  \item \texttt{id} numbers are assigned by Graphlet and are
  guaranteed to be unique within a given graph.

\end{notes}

%
% GT_Graph Attributes
%

\section{\GT{Graph} Attributes}
\label{s:GraphAttributes}

The class \GT{Graph} provides the \texttt{gt} methods as
shortcuts to access the attributes of a graph, node or edge (see
also chapter \ref{c:G++:Attributes}).

\begin{Cdefinition}
  
  \item[GT\_Graph\_Attributes\& GT\_Graph::gt()] \strut\\
  Provides access to the attributes of the graph.
  
  \item[const GT\_Graph\_Attributes\& GT\_Graph::gt() const] \strut\\
  \texttt{const} version of the above method. This is the
  preferred access method.
  
  \item[GT\_Node\_Attributes\& GT\_Graph::gt(node \Param{n})] \strut\\
  Provides access to the attribute of node \emph{n}.
  
  \item[const GT\_Node\_Attributes\& GT\_Graph::gt(node \Param{n}) const]
  \strut\\
  \texttt{const} version of the above method. This is the
  preferred access method.
  
  \item[GT\_Edge\_Attributes\& GT\_Graph::gt(edge \Param{e})] \strut\\
  Provides access to the attribute of edge \emph{e}.
  
  \item[const GT\_Edge\_Attributes\& GT\_Graph::gt(edge
  \Param{e}) const]
  \strut\\
  \texttt{const} version of the above method. This is the
  preferred access method.
  
\end{Cdefinition}

Example \ref{e:C++:AttributesIntro} shows a simple example which 
changes the labels of a graph.  Example \ref{e:C++:stLabels} example 
assigns a label ``s - t'' to each edge in the graph, where \emph{s} is 
the label of the source node and \emph{t} is the label of the target 
node.


\begin{example}{e:C++:AttributesIntro}{A Simple Example for the use of 
Graphlet Attributes}
\begin{verbatim}  
GT_Graph g;

//
// Retrieve the label of a graph
//

const string& l1 = g.gt().label();

//
// Set the label of g to "42"
//

g.gt().label("42");
\end{verbatim}
\end{example}

\begin{example}{e:C++:stLabels}{Assigning ``s - t`` to edge labels in a 
graph}
\begin{verbatim}
GT_Graph g;
edge e;

forall_edges (e,g) {

    // Find source and target nodes

    node source = g.gt(e).source();
    node target = g.gt(e).target();

    // Assemble a new label

    string new_label = g.gt(source).label() + " - " + g.gt(target).label();
    
    // set the new label

    g.gt(e).label (l);
}
\end{verbatim}
\end{example}


%
% Attribute Implementation Notes
%

\subsection{How to detect wether an attribute has changed}
\label{s:DetectingChangesInAttributes}

\Graphlet{}'s attributes are based on the class
\GT{Tagged\_Attributes}.  \GT{Tagged\_Attributes} implements a
mechanism which tracks wether attributes have been
\emph{initialized} or \emph{changed}:

\begin{description}

\item[initialized] is set as soon as a value is assigned to the 
attribut.

\item[changed] indicates that the attribute has changed since the
last draw peration.  \emph{changed} is set whenever the
value of the attribute is changed.  \emph{changed} is reset when
the object (that is, the graph, node, edge or label) is redrawn.

\end{description}

\noindent The following classes are derived from \GT{Tagged\_Attributes}:

\begin{itemize}

  \item \GT{Common\_Attributes}. This class is the base class for
  \GT{Node\_Attributes}, \GT{Edge\_Attributes} and
  \GT{Graph\_Attributes}.

  \item \GT{Common\_Graphics}. This class is the base class for
  \GT{Node\_Graphics}, \GT{Edge\_Graphics} and
  \GT{Graph\_Graphics}.
  
  \item \GT{Node\_NEI}, \GT{Edge\_NEI}. These classes implement
  Graphlet's node and edge anchors.

\end{itemize}


\noindent The class \GT{Tagged\_Attributes} uses a bit set
\footnote{The tags are implemented with the C++ datatype
  \texttt{unsigned} in the current implementation, although this
  might change in the future} to keep track of the state of
attributes.  This set is defined as follows. Let $A$ be the set
of attributes implemented in the class based on
\GT{Tagged\_Attributes}\footnote{Except for graphics, $A$
  consists of (the names of) all attributes as listed in this
  documentation. In the case of graphics, $A$ consists of all
  attributes except \texttt{center} and \texttt{line}, which are
  replaced by \texttt{geometry}}. For each $a \in A$, Graphlet
defines a constant \texttt{tag\_}$a$. Then, the bitsets
\texttt{initialized}, \texttt{changed} and \texttt{updated} are
defined as

\[
\mbox{initialized}, \mbox{changed}
\subseteq
\bigcup_{a \in A} \mathtt{tag\_}a
\]
%Wow, I can do formulas in LaTeX !

\noindent For example, the set of all graphics attributes which affect labels 
can be written in C++ as

\begin{verbatim}
unsigned text_attributes =
    GT_Common_Graphics::tag_anchor |
    GT_Common_Graphics::tag_geometry |
    GT_Common_Graphics::tag_fill |
    GT_Common_Graphics::tag_font |
    GT_Common_Graphics::tag_justify |
    GT_Common_Graphics::tag_stipple |
    GT_Common_Graphics::tag_width;
\end{verbatim}

\noindent The following methods can be used to examine the status of
attributes:

\begin{Cdefinition}
  
  \item[bool is\_initialized (const unsigned attribute) const] \strut\\
  Returns whether \texttt{attribute} has been initialized.  If
  used with a set of attributes, checks wether at least one of
  the attributes is initialized.
  
  \item[unsigned initialized () const] \strut\\
  Returns the set of initialized attributes.

  \item[unsigned nothing\_initialized () const] \strut\\
  Returns \emph{true} if \emph{no} attribute has been
  initialized, and \emph{false} otherwise.
  
  \item[bool is\_changed (const unsigned attribute) const] \strut\\ 
  Returns whether \texttt{attribute} has changed since the last draw 
  operation.  If used with a set of attributes, checks wether at least 
  one of the attributes is changed.

  \item[unsigned changed () const] \strut\\
  Returns the set of attributes which have been changed since the 
  last drawing operation.

  \item[unsigned nothing\_changed ()] \strut\\
  Returns \emph{true} if no attribute has been changed since 
  the last drawing operation, and \emph{false} otherwise.
  
  \item[bool set\_changed (const unsigned attribute) const] \strut\\
  Declares that attribute (which may be a single attribute or a
  set of attributes) has changed since the last drawing
  operation. This means that the attribute will be upated and
  drawn in the next drawing operation.
   
  \item[bool reset\_changed (const unsigned attribute)] \strut\\
  Declares that attribute (which may be a single attribute or a
  set of attributes) has not changed since the last drawing
  operation.  This means that the attribute will \textbf{not} be
  upated and \textbf{not} be drawn in the next drawing operation.
  \emph{Handle with care}.
  
  \item[void all\_changed (unsigned begin, unsigned end)] \strut\\
  This method sets the state of all initialized attributed to
  \emph{changed}.

\end{Cdefinition}


\begin{notes}
  
  \item There are only few situations where
  \texttt{set\_changed}, \texttt{reset\_changed} and
  \texttt{all\_changed} need to be used, and all of them are
  related to speed optimization. Handle with care unless you are
  in the mood for an extended debugging session.

\end{notes}




%
% Drawing
%

\section{Drawing Graphs}


This section explains how to use \GT{Graph}'s \texttt{draw}
methods to draw a graph or a portion of a graph explicitly.
Algorithms which use the class \verb|GT_Tcl_Algorithm<>| and do
\emph{not} perform algorithm animation, do not need to call
\texttt{draw} explicitly. Implementore of such algorithms can
gladly skip this section.

Graphlet does not automatically redraw a graph whenever an
attribute is changed\footnote{If we would do that, the
  performance penalty would almost be the death-of-the-system
  penalty.}.  Instead, the programmer must explicitly draw the
graph.  \texttt{GT}\_Graph provides the following methods to draw
graphs:

\begin{Cdefinition}

  \item[int GT\_Graph::draw()]
  Draws the whole graph on all drawing areas which are associated with 
  the graph.

  \item[int GT\_Graph::draw (node \Param{n})]
  Draws the node \emph{n} on all drawing areas which are associated with 
  the graph.

  \item[int GT\_Graph::draw (edge \Param{e})]
  Draws the edge \emph{e} on all drawing areas which are associated with 
  the graph.

\end{Cdefinition}

\begin{example}{e:drawExample}{Move all nodes by 10 pixels}
\begin{verbatim}
forall_nodes (n, g.leda()) {
    double x = g.gt(n).graphics()->x();
    double y = g.gt(n).graphics()->y();
    g.gt(n).graphics()->x (x+10);
    g.gt(n).graphics()->y (y+10);
}

g.draw(); 
\end{verbatim}
\end{example}

\noindent Example \ref{e:drawExample} shows how to use the
\texttt{draw} method.  In general, the method \texttt{draw()} is
less efficient than \texttt{draw(node)} or \texttt{draw(edge)},
since \texttt{draw()} has to search the whole graph for objects
which must be updated.  However, it is usually a good idea to
call \texttt{draw()} at the end of an algorithm unless the
following three points apply:

\begin{itemize}
  \item The application is time critical,
  
  \item The programmer exactly knows which changes have occurred,
  and updates them manually with \texttt{draw(node)} and
  \texttt{draw(edge)} operations.
  
\end{itemize}


%
% Performance Issues
%

\subsection{Performance Issues}

Drawing graphs is actually a lot more than just translating
coordinates into graphics operations. For example, if the
coordinates of a node have changed, then the coordinates of its
label, all its adjacent edges and probably their labels must be
adjusted. If an algorithm updates coordinates step-by-step, this
would result in many too many updates.

Therefore, unless algorithm animation is intended, the
\texttt{draw} methods should only be called at the end of
an algorithm. 


%
%
%

\subsection{Animation}

\GT{Graph} supports animation out of the box; all the user has
to do are the following two steps:

\begin{itemize}
  
  \item Call \texttt{draw} at the point where the graph on the
  screen should be updated.
  
  \item Call the \textbf{Tcl command} \texttt{update} to update the
  screen, e.g.
  \begin{verbatim}
Tcl_Interp* interp = tcl_interpreter();

int code = Tcl_Eval (interp, "update");
if (code == TCL_ERROR) {
    return TCL_ERROR;
}
  \end{verbatim}

  where \texttt{interp} is the current Tcl interpreter. One way to
  access the interpreter is through the method \GTTcl{info::interp()}.

\end{itemize}

\begin{notes}
  
  \item It is usually easier to implement algorithm animation
  with GraphScript than from C++.
  
  \item Contrary to common wisdom, the method \GT{Graph::update}
  is \textbf{not} a wrapper for evaluating the Tcl command
  \texttt{update}.

\end{notes}


%%%%%%%%%%%%%%%%%%%%%%%%%%%%%%%%%%%%%%%%%%
%
% Graphlet Attributes
%
%%%%%%%%%%%%%%%%%%%%%%%%%%%%%%%%%%%%%%%%%%


\chapter{Graphlet Attributes}
\label{c:Attributes}

The attributes in Graphlet can roughly be divided in the following 
categories:

\begin{itemize}

  \item Common attributes for graphs, nodes and edges (Section
  \ref{s:Attributes:Common})

  \item Graph specific attributes (Section
  \ref{s:Attributes:Graph})

  \item Node specific attributes (Section
  \ref{s:Attributes:Node})

  \item Edge specific attributes (Section \ref{s:Attributes:Edge})
  
  \item Graphics attributes (Section
  \ref{s:Attributes:Graphics}). Graphs, nodes and edges share
  \emph{one} data structure for this.

\end{itemize}

Most attributes in the C++ interface are also available in
GraphScript, and have a corresponding GML key. We have tried to
keep names of these attributes and their values are kept as
similar as possible.

%
% Data Structure
%

\section{Data Structure}


%
% Common Attributes
%

\section{Common Attributes}
\label{s:Attributes:Common}

The following attributes are common with graphs, nodes and edges:

\begin{CAttributes}

  \item[int id] \strut\\
  The \texttt{id} is a number which is assigned to an object.
  Note that node and edge \texttt{id}'s are only unique within
  their graph. Nodes in different graphs may have the same id.
  
  \item[string label] \strut\\
  This is the textual label of the graph, node or edge. Currently,
  graph labels are not displayed.
  
  \item[\GT{Key} label] \strut\\
  See the sections on node \ref{s:Attributes:Node} and edge
  attributes \ref{s:Attributes:Edge}.
  
  \item[int uid] \strut\\
  This is the \emph{user interface id} of the object (graph,
  node, edge) as used by the Tcl interface (see also \ref{unknown}).
  \emph{This value is initialized by Graphlet and should not be changed.}

  \item[int label\_uid] \strut\\
  This is the \emph{user interface id} of the label (graph,
  node, edge) as used by the Tcl interface (see also \ref{unknown}).
  \emph{This value is initialized by Graphlet and should not be changed.}

\end{CAttributes}


%
% Graph specific Attributes
%

\section{Graph specific Attributes}
\label{s:Attributes:Graph}

The following attributes are specific for graphs:


\begin{CAttributes}

  \item[int version] \strut\\
  Reserved for future use.  The corresponding GML attribute is 
  \texttt{Version}).
  
  \item[string creator] \strut\\
  The program which created the graph (default is ``Graphlet'').
  
  \item[GT\_Graph\_Graphics* graphics] \strut\\
  Pointer to the graphical description of the graph.  For details, see 
  the Section \ref{s:Attributes:Graphics}.
  
  \item[GT\_Graph\_Graphics* label\_graphics] \strut\\
  Pointer to the graphical description of the label of the graph.  For 
  details, see Section \ref{s:Attributes:Graphics}.
  
\end{CAttributes}


%
% Node specific Attributes
%

\section{Node specific Attributes}
\label{s:Attributes:Node}

The following attributes are specific for nodes:

\begin{CAttributes}
  
  \item[\GT{Graph}* g] \strut\\
  Pointer to the graph.

  \item[edge e] \strut\\
  Pointer to the edge.
  
  \item[GT\_Node\_Graphics* graphics] \strut\\
  Pointer to the graphical description of the node.  For details,
  see Section \ref{s:Attributes:Graphics}.
  
  \item[GT\_Node\_Graphics* label\_graphics] \strut\\
  Pointer to the graphical description of the label of the node.
  For details, see Section \ref{s:Attributes:Graphics}.
  
  \item[\GT{Key} label\_anchor] \strut\\
  This controls where the label is placed.  Values are:

  \begin{ttdescription}
    \item[GT\_Keys::anchor\_center] \strut\\
    The label is placed at the center of the bounding rectangle of the 
    node.

    \item[GT\_Keys::anchor\_n] \strut\\
    The label is placed at the center of the top side of the 
    bounding rectangle of the node.

    \item[GT\_Keys::anchor\_ne] \strut\\
    The label is place in the upper right corner of the bounding 
    rectangle of the node.

    \item[GT\_Keys::anchor\_e] \strut\\
    The label is placed at the center of the right side of the 
    bounding rectangle of the node.

    \item[GT\_Keys::anchor\_se] \strut\\
    The label is place in the lower right corner of the bounding 
    rectangle of the node.

    \item[GT\_Keys::anchor\_s] \strut\\
    The label is placed at the center of the bottom side of the 
    bounding rectangle of the node.

    \item[GT\_Keys::anchor\_sw] \strut\\
    The label is place in the lower left corner of the bounding 
    rectangle of the node.

    \item[GT\_Keys::anchor\_w] \strut\\
    The label is placed at the center of the left side of the 
    bounding rectangle of the node.

    \item[GT\_Keys::anchor\_nw] \strut\\
    The label is place in the upper left corner of the bounding 
    rectangle of the node.
    
  \end{ttdescription}

  \begin{notes}
    \item Note that the placement of a label is also affected by
    the attributes \texttt{anchor} and \texttt{justify} of the
    graphics information which is associated with the label. See
    section \ref{s:Attributes:CommonGraphics} for details.
  \end{notes}
  
  \item[\GT{Node\_NEI}*, node\_nei] \strut\\
  Pointer to the node/edge interface structure which controls how
  edges are attached to their endnodes. Ask Walter Bachl,
  \url{bachl@fmi.uni-passau.de} for details.

\end{CAttributes}



%
% Edge specific Attributes
%

\section{Edge specific Attributes}
\label{s:Attributes:Edge}

The following attributes are specific for edges:

\begin{CAttributes}
  
  \item[\GT{Graph}* g] \strut\\
  Pointer to the graph.

  \item[edge e] \strut\\
  Pointer to the edge.

  \item[node source] \strut\\
  Pointer to the source node of the edge.  \emph{This attribute is
    managed by \GT{Graph} and must not be changed.}
   
  \item[node target] \strut\\
  Pointer to the target node of the edge. \emph{This attribute is
    managed by \GT{Graph} and must not be changed.}
  
  \item[GT\_Edge\_Graphics* graphics] \strut\\
  Pointer to the graphical description of the edge.  For details, see 
  Section \ref{s:Attributes:Graphics}.
  
  \item[GT\_Edge\_Graphics* label\_graphics] \strut\\
  Pointer to the graphical description of the label of the edge.  For 
  details, see Section \ref{s:Attributes:Graphics}.  
  
  \item[\GT{Key} label\_anchor] \strut\\
  This controls where the label is placed.  Values are:

  \begin{description}
    \item[\GT{Keys::anchor\_first}] \strut\\
    Attach the label to the first segment of the edge.
    \item[\GT{Keys::anchor\_center}] \strut\\
    Attach the label to the middle segment of the edge.
    \item[\GT{Keys::anchor\_last}] \strut\\
    Attach the label to the last segment of the edge.
  \end{description}  


  \begin{notes}
    
    \item In all cases, the label is attached at the center of
    the corresponding segment.

    \item Note that the placement of a label is also affected by
    the attributes \texttt{anchor} and \texttt{justify} of the
    graphics information which is associated with the label. See
    section \ref{s:Attributes:CommonGraphics} for details.

  \end{notes}
  
  \item[\GT{Edge\_NEI}* edge\_nei] \strut\\
  Pointer to the node/edge interface structure which controls how
  edges are attached to their endnodes. Ask Walter Bachl,
  \url{bachl@fmi.uni-passau.de} for details.

\end{CAttributes}



%
% Graphics Attributes
%

\section{Graphics Attributes}
\label{s:Attributes:Graphics}

The \Graphlet{} attributes for graphs, nodes and edges do not
contain information on the graphical appearance of graphs, nodes
and edges.  Instead, the graphical information is stored in
separate classes \GT{Graph\_Graphics}\footnote{The classes
  \GT{Graph\_Graphics}, \GT{Node\_Graphics} and
  \GT{Edge\_Graphics} are currently dummies and are provided for
  future extensions.}, \GT{Node\_Graphics} and
\GT{Edge\_Graphics}, which are all based on the class
\GT{Common\_Graphics}.  With that, the graphical appearance of
graphs, nodes, edges and their label is described and handled
through a common, device independend data structure.

The classes \GT{Graph\_Attributes}, \GT{Node\_Attributes} and 
\GT{Edge\_Attributes} contain pointers to these graphical 
descriptions.  Per convention, these pointers are accessed through the 
methods \texttt{graphics} and \texttt{label\_graphics}.

\begin{notes}
  
  \item Graphlet does not guarantee that these pointers are
  \textbf{not} \texttt{0}. Instead, every procedure which uses
  graphics information must test wether the graphics are not
  \texttt{0}.

  \item The correct way to initialize graphics information is as follows:
\begin{verbatim}
if (g.gt().graphics() == 0) {
    g.gt().graphics (g.new_graph_graphics());
}
\end{verbatim}
  
  \item Done use \verb|new GT_Graph_Graphics| instead of
  \verb|g.new_graph_graphics()| in the above program; as this
  will override customization options.

\end{notes}


\subsection{How to access a graphics attribute}

The following C++ constructs retrieve the value of a graphics 
attribute:

\begin{ttdescription}
  
  \item[\Param{g}.gt().graphics()-$>$\Param{a}()] \strut\\
   Access attribute \emph{a} of graph \texttt{g}.
  
  \item[\Param{g}.gt(n).graphics()-$>$\Param{a}()] \strut\\
  Access attribute \emph{a} of node \texttt{n} in graph \texttt{g}.
  
  \item[\Param{g}.gt(e).graphics()-$>$\Param{a}()] \strut\\
  Access attribute \emph{a} of edge \texttt{e} in graph \texttt{g}.
  
  \item[\Param{g}.gt().label\_graphics()-$>$\Param{a}()] \strut\\
  Access attribute \emph{a} of the label of graph \texttt{g}.
  
  \item[\Param{g}.gt(n).label\_graphics()-$>$\Param{a}()] \strut\\
  Access attribute \emph{a} of the label of node \texttt{n} in graph 
  \texttt{g}.
  
  \item[\Param{g}.gt(e).label\_graphics()-$>$\Param{a}()] \strut\\
  Access attribute \emph{a} of the label of edge \texttt{e} in graph 
  \texttt{g}.

\end{ttdescription}


The following C++ constructs are used to set or change the value
of a graphics attribute:

\begin{ttdescription}
  
  \item[\Param{g}.gt().graphics()-$>$\Param{a} (\Param{new\_a});]
  \strut\\
  Set graphics attribute \Param{a} of the label of graph
  \Param{g} to the value \Param{new\_a}.
    
  \item[\Param{g}.gt(\Param{n}).graphics()-$>$\Param{a} (\Param{new\_a});]
  \strut\\
  Set graphics attribute \Param{a} of the label of node \Param{n}
  in graph \Param{g} to the value \Param{new\_a}.
  
  \item[\Param{g}.gt(\Param{e}).graphics()-$>$\Param{a} (\Param{new\_a});]
  \strut\\
  Set graphics attribute \Param{a} of the label of edge \Param{e}
  in graph \Param{g} to the value \Param{new\_a}.
  
  \item[\Param{g}.gt().label\_graphics()-$>$\Param{a} (\Param{new\_a});]
  \strut\\
  Set graphics attribute \Param{a} of the label of graph \Param{g} 
  to the value \Param{new\_a}.
  
  \item[\Param{g}.gt(\Param{n}).label\_graphics()-$>$\Param{a} (\Param{new\_a});]
  \strut\\
  Set graphics attribute \Param{a} of the label of node \Param{n} in 
  graph \Param{g} to the value \Param{new\_a}.
  
  \item[\Param{g}.gt(\Param{e}).label\_graphics()-$>$\Param{a} (\Param{new\_a});]
  \strut\\
  Set graphics attribute \Param{a} of the label of edge \Param{e} in 
  graph \Param{g} to the value \Param{new\_a}

\end{ttdescription}


%
% Common Graphics Attributes
%

\subsection{Common Graphics Attributes}
\label{s:Attributes:CommonGraphics}

\Graphlet{}'s class \GT{Common\_Graphics} implements a universal,
device independent description of the graphical appearance of an
object. Common graphics attributes describe the appearance of
graphs, nodes, edges and their labels.  The graphical attributes
are based on the graphical attributes used in the Tk toolkit.

The following list of graphics attributes implemented with
Graphlet.  All these attributes are modelled after the
corresponding Tk canvas attributes.  For a detailed discussion,
see the Tk documentation on canvases.

\begin{CAttributes}

  \item[GT\_Key type] \strut\\
  This is the Tk item type of the object.  \Graphlet{} supports all Tk 
  item types except window.  The following values are valid types:
  \begin{ttdescription}
    \item[GT\_Keys::type\_arc] \strut\\
    The Tk graphics object type \texttt{arc}.
    \item[GT\_Keys::type\_bitmap] \strut\\
    The Tk graphics object type \texttt{bitmap}.
    \item[GT\_Keys::type\_image] \strut\\
    The Tk graphics object type \texttt{image}.
    \item[GT\_Keys::type\_line] \strut\\
    The Tk graphics object type \texttt{line}.
    \item[GT\_Keys::type\_oval] \strut\\
    The Tk graphics object type \texttt{oval}.
    \item[GT\_Keys::type\_polygon] \strut\\
    The Tk graphics object type \texttt{polygon}.
    \item[GT\_Keys::type\_rectangle] \strut\\
    The Tk graphics object type \texttt{rectangle}.
    \item[GT\_Keys::type\_text] \strut\\
    The Tk graphics object type \texttt{text}.
  \end{ttdescription}
  
  \begin{notes}
    \item It is legal to change the type of a graphics object after it has 
    been initialized.  However, special precaution has to be taken to 
    keep the selection up to date.
    \emph{Documentation of this will be extended as soon as the API is 
      ready}. \ToDo{}.
    \item The current implementation of Graphlet assumes that the type 
    \emph{text} is only used for labels.
    \item The current implementation of Graphlet assumes that
    edges always have the type \texttt{line}.
    \item The current implementation of Graphlet assumes that
    edges always have the type \texttt{text}.
  \end{notes}
  
  \item[GT\_Rectangle center] \hfill
  {\footnotesize arc, bitmap, image, oval, polygon, rectangle, \emph{text}} \\
  This is a rectangle which describes the center of an object and
  its width and height.

  \begin{notes}
    \item
    The geometry of \emph{line} and \emph{polyline} objects is described 
    by the \texttt{line} attribute, and \texttt{center} is undefined for 
    them.
    \item The \emph{width} and \emph{height} of \emph{text} objects
    are undefined, as this will be calculated and set by the
    windowing system.
  \end{notes}
      
  \item[GT\_Polyline line] \hfill
  {\footnotesize line, polyline}\\
  This is a list of points which describes an object of type 
  \emph{line} or \emph{polygon}.

  \item[GT\_Key fill] \hfill
  {\footnotesize arc, line, polygon, oval, rectangle, text} \\
  The color with which an object is filled.
  
  \item[GT\_Key outline] \hfill
  {\footnotesize arc,oval,rectangle} \\
  The color with which the outline of an object is drawn.
  \begin{notes}
  \item Tk uses the \texttt{fill} attribute to specify the color of a 
  line.
  \item Tk polygons do not have an outline.
  \end{notes}

  \item[GT\_Key stipple] \hfill
  {\footnotesize arc, line, polygon, oval, rectangle, text} \\
  The name of a Tk bitmap which is used as a stipple to draw the 
  object.
      
  \item[GT\_Key anchor] \hfill
  {\footnotesize bitmap, image, text} \\
  The Tk anchor of an object.
      
  \item[double width] \hfill
  {\footnotesize arc, line, oval, rectangle, text} \\
  If the object is a line, then \texttt{width} of that line.  
  Otherwise, \texttt{width} spoecifies the width of the \emph{outline} 
  of an object.

  \item[double extent] \hfill
  {\footnotesize arc} \\
  \texttt{extent} is the length of the arc, 
  in \emph{degrees}. For example, 90 means a quarter circle, 180 means 
  a half circle, and 270 is a three quarter circle.
      
  \item[double start] \hfill
  {\footnotesize arc} \\
  \texttt{starty} is the start of the 
  arc, in degrees.  For example, \texttt{start} 0 and \texttt{extent} 
  180 are the right half of a circle, and \texttt{start} 90 and 
  \texttt{extent} 180 are the lower half of a circle.

  \item[GT\_Key style] \hfill
  {\footnotesize arc} \\
  Implements the style of an \emph{arc} item. Valid values are
  \begin{itemize}
  \item \GT{Keys::style\_pie}
  \item \GT{Keys::style\_chord} 
  \item \GT{Keys::style\_slice}
  \end{itemize}
  
  \item[GT\_Key background] \hfill
  {\footnotesize bitmap} \\
  Sets the background color of a bitmap.

  \item[GT\_Key foreground] \hfill
  {\footnotesize bitmap} \\
  Sets the foreground color of a bitmap.
      
  \item[GT\_Key bitmap] \hfill
  {\footnotesize bitmap} \\
  The name of the Tk bitmap.
  \begin{note}
  Use Tcl/Tk's \texttt{bitmap} command to create a bitmap.
  \end{note}
      
  \item[GT\_Key image] \hfill
  {\footnotesize image} \\
  The name of the Tk image.
  \begin{note}
  Use Tcl/Tk's \texttt{image} command to create an image.
  \end{note}
      
  \item[GT\_Key arrow] \hfill
  {\footnotesize line} \\
  Determines whether a line has an arrow. Valid values are
  \begin{itemize}
  \item \GT{Keys::arrow\_none}
  \item \GT{Keys::arrow\_first}
  \item \GT{Keys::arrow\_last}
  \item \GT{Keys::arrow\_both} 
  \end{itemize}
  \begin{notes}   
  \item The arrow is usually managed by \Graphlet{}.  The default is 
  to use \GT{Keys::arrow\_none} undirected graphs and 
  \GT{Keys::arrow\_last} in directed graphs.
  \item Other values than the defaul are possible, but might produce to 
  strange and probably misleading drawings.
  \item When a graph changes from directed to undirected or vice 
  versa, Graphlet resets the arrow.
  \end{notes} 

  \item[GT\_Key arrowshape] \hfill
  {\footnotesize line} \\
  \emph{Currently not implemented.}
      
  \item[GT\_Key capstyle] \hfill
  {\footnotesize line} \\
  Controls how the endpoints of lines are drawn.  See the Tk 
  documentation for details. \emph{Valid values to be determined.}
      
  \item[GT\_Key joinstyle] \hfill
  {\footnotesize line} \\
  Controls how the joints of lines are drawn.  See the Tk 
  documentation for details. \emph{Valid values to be determined.}
      
  \item[bool smooth] \hfill
  {\footnotesize line, polygon} \\
  Controls whether a line or polygon is drawn as a spline 
  (\texttt{true}) or as a straight line segments (\texttt{false}). 
  The default is \texttt{false}, which means that no splines are used.
  \begin{note}
  Most layout algorithms expect lines to be drawn as straight lines 
  and do not account for splines.
  \end{note}
      
  \item[int splinesteps] \hfill
  {\footnotesize line, polygon} \\
  Number of interpolation steps for splines.  The dewfault is 0.  See 
  the Tk documentation for details.
      
  \item[GT\_Key justify] \hfill
  {\footnotesize text} \\
  Justification of text. \emph{Valid values to be determined.}
      
  \item[GT\_Key font] \hfill
  {\footnotesize text} \\
  \emph{Currently not implemented, as the support for fonts is delayed 
  as Tk will implement a new, more portable font scheme in future 
  versions.}
        
\end{CAttributes}


\begin{notes}

  \item The following colors are predefined:
  \begin{itemize}
  \item \GT{Keys::white}
  \item \GT{Keys::black}
  \item \GT{Keys::red}
  \item \GT{Keys::green}
  \item \GT{Keys::blue}
  \end{itemize}
  
  \item
  A new color should have the format \verb|#RRGGBB|, where 
  \texttt{RR}, \texttt{GG} and \texttt{BB} are hexadecimal numbers 
  which specify the values for the red, green and blue values of a 
  color. To add a new color, use the following schema:
  \begin{alltt}
  GT_Key silly\_color = graphlet->keymapper.add ("#123456");
  \end{alltt}
  
  \item 
  The class \GT{Common\_Attributes} holds attributes for \emph{all} 
  types of graphical objects.  But, only the attributes which apply 
  for the current \texttt{type} are used in the display.  That is, a 
  node which is displayed as a rectangle may have \texttt{image} set 
  although this is not used.  When the \texttt{type} is switched to 
  \texttt{image}, the information stored in \texttt{image} is used.  
  Nevertheless, the API allows that all attributes may be changed or 
  retrieved at any time, regardless of type.

  \item
  Graphlet stores all \emph{initialized} graphics and label graphics 
  attribute in GML files. They are stored under the keys
  \texttt{graphics} and \texttt{LabelGraphics}.

\end{notes}



%
% Shortcuts for coordinates and size
%

\subsection{Shortcuts for coordinates and size}

The class \GT{Common\_Graphics} provides the
following shortcuts for coordinates and size of an object:

\begin{Cdefinition}
  
  \item[double x () const] \strut\\
  Shortcut for accessing the \texttt{x} coordinate of the center of 
  the object.
  
  \item[void x (const double x)]  \strut\\
  Shortcut for accessing the \texttt{x} coordinate of the center of 
  the object.
  
  \item[double y () const] \strut\\
  Shortcut for accessing the \texttt{y} coordinate of the center of 
  the object.

  \item[void y (const double y)] \strut\\
  Shortcut for accessing the \texttt{y} coordinate of the center of 
  the object.
  
  \item[double w () const] \strut\\
  Shortcut for accessing the width of the object.

  \item[void w (const double w)] \strut\\
  Shortcut for accessing the width of the object.
  
  \item[double h () const] \strut\\
  Shortcut for accessing the height of the object.

  \item[void h (const double h)] \strut\\
  Shortcut for accessing the height of the object.
  
\end{Cdefinition}

Example \ref{e:C++:MaximumWidthOfAllNodes} shows how to use these 
shortcuts.

\begin{example}%
{e:C++:MaximumWidthOfAllNodes}%
{Computing the maximum width of all nodes in a graph}
\begin{verbatim}
double Sample_class::compute_maximum_width (const GT_Graph& g)
{
    double max = 0; // Minimum legal value for a width
    
    node n;
    forall_nodes (n, g.leda()) {
        if (g.gt(n).graphics() != 0) {
            double w = g.gt(n).graphics()->w();
            if (max < w) {
                w = max;
            }
        }
    }
    
    return max;
}
\end{verbatim}
\end{example}



%%%%%%%%%%%%%%%%%%%%%%%%%%%%%%%%%%%%%%%%%%
%
%  GT_Algorithm
%
%%%%%%%%%%%%%%%%%%%%%%%%%%%%%%%%%%%%%%%%%%

\chapter{Implementing Graph Algorithms}
\label{c:Algorithms}

\CSourceCode{}{base}{Algorithm}

The class \GT{Algorithm} provides a base class and a standard
interface for the implementation of graph algorithms. The key
idea is that each algorithm is a class, and is invoked with the
\texttt{run} method. Parameters and results are stored as class
member variables (instead of parameters and/or function results).

While this is obviously not the only sane way to implement
algorithms, its major advantage is that all of Graphlets
interfaces provide special support for algorithms which are
implemented with this class. Especially, the Tcl interface for
algorithms (see Chapter \ref{c:TclInterface}) is designed to
take an algorithm which has been implemented with \GT{Algorithm}
and construct a Tcl interface for it on the fly -- all the
programmer has to supply is code for parsing parameters and error
handling.

Another advantage of the above concept is that since algoithms
are objects, it is easy to have several algorithms with
\emph{different parameter sets} around. Actually, the value of a
parameter stays the same until it is reset. Also, algorithms can
be passed anonymously as \emph{parameters}, a construct which is
not easy to realize otherwise in C++.


%
% The class GT_algorithm
%

\section{The class \GT{Algorithm}}

All algorithms in \Graphlet{} should be derived from the class
\GT{Algorithm}. Example \ref{e:C++:SampleAlgorithmClass} shows a
minimal example for an algorithm class implemented with
\GT{Algorithm}.

\begin{example}%
{e:C++:SampleAlgorithmClass}%
{A Sample Algorithm Class Declaration}
\begin{verbatim}
class Sample : public GT_Algorithm {
    public:
    Sample (const string& name);
    virtual ~Sample ();
    virtual int run (GT_Graph& g);
    virtual int check (GT_Graph& g, string& message);
}
\end{verbatim}
\end{example}


%
% Methods which must be provided
%

\subsection{Methods which must be provided by the derived class}

\begin{Cdefinition}

  \item[Sample (const string\& \Param{name})] \strut\\
  This is the constructor of the class \texttt{Sample}.  Its parameter 
  should be the name of the command.  \GT{Algorithm}'s constructor 
  takes a single argument, which is the \emph{name} of the algorithm, 
  and should be defined as follows:

\begin{alltt}
Sample::Sample(const string& name) : GT_Algorithm (name) \{
    \textnormal{\emph{Initialize the algorithm's parameters here}}
\}
\end{alltt}

  Usually, the name of the algorithm will later be used as the
  name of the associated Tcl command.

  \item[virtual \~ Sample()] \strut\\
  The class \texttt{Sample} must contain a virtual destructor since 
  \GT{Algorithm} already has virtual functions.
  
  \item[virtual int run (GT\_Graph\& \Param{g})] \strut\\
  The method \texttt{run} executes the algorithm.  Its parameter is 
  the input graph.  This procedure should return \GT{OK} if the 
  command could be completed successfully, and \GT{ERROR} otherwise.
  
  \item[virtual int check(GT\_Graph\& \Param{g}, string\& \Param{message})]
  \strut\\
  The method \texttt{check} should be called before \texttt{run} to 
  check whether the current graph \emph{g} is applicable for the 
  algorithm.  This method should return \GT{OK} if the graph is 
  applicable, and \GT{ERROR} otherwise.  The parameter \emph{message} 
  should contain a detailed error message if the check fails.

  \GT{Algorithm} does not automatically call \texttt{check}
  before \texttt{run}. The Tcl interface for \GT{Algorithm} (see
  Section \ref{s:GT_Tcl_Interface}) enforces this. The reason is
  performance, and the fact that Tcl allows for a better error
  checking with its \texttt{catch} statements.
  
\end{Cdefinition}


%
% Other methods
%

\subsection{Other methods which may be overridden by the derived class}

\begin{Cdefinition}
  
  \item[virtual void reset ()] \strut\\
  The method \texttt{reset} should reset all options and
  parameters of the algorithm.
  
  Similarily to \texttt{check}, \texttt{reset} is never called
  automatically.  However, the Tcl interface has a configurable
  option to to do this each time the command is executed (see
  Section \ref{s:GT_Tcl_Interface} for details).

\end{Cdefinition}


%
% Methods provided
%

\section{Methods which are provided by the class \GT{Algorithm}}

\begin{Cdefinition}
  
  \item[static edge find\_self\_loop (GT\_Graph\& \Param{g})] \strut\\
  \texttt{find\_self\_loop} searches for a self look in
  \texttt{g}. It returns \texttt{nil} if no self loop is found.
  
  \item[static bool remove\_all\_bends (GT\_Graph\& \Param{g})] \strut\\
  \texttt{remove\_all\_bends} removes all bends in the edges of \emph{g}.
  It returns \texttt{true} if bends are found, and \texttt{false} 
  otherwise.

  \item[void adjust\_coordinates (GT\_Graph\& \Param{g},
  double \Param{min\_x} = 0,
  double \Param{min\_y} = 0)]
  \strut\\
  \texttt{adjust\_coordinates} moves the graph so that the minimum 
  coordinates are at (min\_x,min\_y).  

\end{Cdefinition}


%
% Example
%

\section{A complete Example}
\label{s:Algorithm:Example}


\subsection{Header}
\label{s:Algorithm:Example:Header}

The following C++ header snippet shows how to implement a
planarity test algorithm\footnote{Well, rather the interface for
  a planarity test algorithm. The real algorithm is said to be
  easy and therefore left as an exercise to the reader.}.  We use
the following conventions for the interface:

\begin{itemize}
  
  \item The result of the algorithm is stored in a variable
  \texttt{is\_planar}.
  
  \item If an error is detected in \texttt{check} (in this case,
  a self loop), then the offending object is stored in a
  variable. This is always a good idea because it might help the
  user interface to provide a cleaner error message.

  \item Options realized as member variables.

\end{itemize}


\begin{verbatim}
class GT_Planarity_Test_Algorithm : public GT_Algorithm {

    GT_CLASS (GT_Planarity_Test_Algorithm, GT_Algorithm);

    // The result is stored in this variable.
    GT_VARIABLE (bool, is_planar);

    // Precondition for LEDA's planarity test is the absence of
    // self loops. If we find a self loop, return it in this
    // variable.
    GT_VARIABLE (edge, self_loop);

    // Options: find kuratowski subgraph
    GT_VARIABLE (bool, find_kuratowski_subgraph);

    // The edges of the kuratowski subgraph are stored here.
    GT_COMPLEX_VARIABLE (list<edge>, kuratowski_edges);
    
public:

    // Constructor and Destructor
    GT_Planarity_Test_Algorithm (const string& name);
    virtual ~GT_Planarity_Test_Algorithm ();

    // Standard methods
    virtual void reset ();
    virtual int run (GT_Graph& g);
    virtual int check (GT_Graph& g, string& message);
};
\end{verbatim}



\subsection{Implementation}
\label{s:Algorithm:Example:Implementation}

The following code implements constructor and destructor for the
planarity test algorithm. Note the conservative treatment of
options: searching for a kuratowski graph is turned off by
default. The constructor is empty in many algorithm classes; 

\begin{verbatim}
GT_Planarity_Test_Algorithm::GT_Planarity_Test_Algorithm (const string& name) :
        GT_Algorithm (name)
{
    the_is_planar = false;
    the_self_loop = 0;
    the_find_kuratowski_subgraph = false;
}


GT_Planarity_Test_Algorithm::~GT_Planarity_Test_Algorithm ()
{
    // Nothing left to do.
}
\end{verbatim}

\noindent The method \texttt{run} is implemented in a straightforward way.
Depending on the \emph{kuratowski\_edges} option, it calls LEDA's
\texttt{PLANAR} algorithm with a different parameter set. Note
again that the result is stored in the member variable
\texttt{is\_planar}. The method always returns \texttt{GT\_OK},
as nothing can go wrong with a planarity test (provided that
\texttt{check} was successful).

\begin{verbatim}
int GT_Planarity_Test_Algorithm::run (GT_Graph& g)
{
    graph& leda = g.leda();

    if (the_find_kuratowski_subgraph) {
        is_planar (PLANAR(leda, the_kuratowski_edges));
    } else {
        the_kuratowski_edges.clear();
        is_planar (PLANAR(leda));
    }

    return GT_OK;
}
\end{verbatim}  


\noindent The method \texttt{check} searches for self loops in the graph.
If it finds one, it (a) deposits it in the member variable
\texttt{self\_loop}, (b) sets \texttt{message} to a description
of the error and (c) returns \GT{ERROR}.

\begin{verbatim}
int GT_Planarity_Test_Algorithm::check (GT_Graph& g, string& message)
{
    edge e = GT_Algorithm::find_self_loop(g);
    if (e != nil) {
        message = "graph contains self loop";
        self_loop (e);
        return GT_ERROR;
    } else {
        self_loop (0);
        message = "";
        return GT_OK;
    }
}
\end{verbatim}

\noindent The method \texttt{reset} is optional and resets all options and
member variables:

\begin{verbatim}
void GT_Planarity_Test_Algorithm::reset ()
{
    the_is_planar = false;
    the_self_loop = 0;
    the_find_kuratowski_subgraph = false;
    the_kuratowski_edges.clear();
}
\end{verbatim}



%%%%%%%%%%%%%%%%%%%%%%%%%%%%%%%%%%%%%%%%%%
%
% The Tcl interface
%
%%%%%%%%%%%%%%%%%%%%%%%%%%%%%%%%%%%%%%%%%%


\chapter{The Tcl interface}
\label{c:TclInterface}


%
% The class GT_Tcl
%

\section{The class \GT{Tcl}}
\label{s:GT_Tcl}

\CSourceCode{}{tcl}{Tcl}

The class \GT{Tcl} provides several utilities to ease the Tcl
- \Graphlet{} interface.


% C++ -> Tcl converters

\subsection{Tools for converting C++ objects into Tcl structures}

\begin{Cdefinition}

  \item[static string list\_of\_strings (const list$<$string$>$\& \Param{s})]
  Constructs a Tcl list of the strings in \emph{s}.

  \item[static void list\_of\_strings (const list$<$string$>$\& \Param{s},
  string\& \Param{result})]
  Fills \emph{result} with  a Tcl list of the strings in \emph{s}.

\end{Cdefinition}


% GT --> Tcl converters

\subsection{Tools for converting GT objects into Tcl objects}

GraphScript does not directly represent graphs, nodes and edges
as Tcl objects; instead, \emph{identifiers} are introduced for
them which are then used to address graphs, nodes and edges from
GraphScript. generally, the form of such an identifier is

\begin{quote}
  \texttt{GT:}\emph{uid} resp.\ \texttt{GT:}\emph{label\_uid}
\end{quote}

where \emph{uid} resp.\ \emph{label\_uid} are the values
\emph{user interface ids} as described in Section
\ref{s:Attributes:Common}. The following functions can be
used to convert nodes and edges to their Graphlet identifiers:

\begin{Cdefinition}
  
  \item[static string gt (const \GT{Graph}\& \Param{g})]
  \strut\\
  Returns the \GraphScript{} identifier of graph \emph{g}.
  
  \item[static string gt (const \GT{Graph}\& \Param{g}, const node \Param{n})]
  \strut\\
  Returns the \GraphScript{} identifier of node \emph{n}
  in graph \emph{g}.
  
  \item[static string gt (const \GT{Graph}\& \Param{g}, const edge \Param{e})]
  \strut\\
  Returns the \GraphScript{} identifier of edge \emph{e}
  in graph \emph{g}.

  \item[static string list\_of\_nodes (const \GT{Graph}\& \Param{g},
  const list<node>\& \Param{nodes});]
  \strut\\
  Constructs a Tcl list of the nodes in \emph{nodes} and returns it.
  
  \item[static void list\_of\_nodes (const \GT{Graph}\& \Param{g},
  const list<node>\& \Param{nodes},
  string\& \Param{result});]
  \strut\\
  Constructs a Tcl list of the nodes in \texttt{nodes} and
  returns it in \emph{result}.

  \item[static string list\_of\_edges (const \GT{Graph}\& \Param{g},
  const list<edge>\& \Param{edges});]
  \strut\\
  Constructs a Tcl list of the edges in \emph{edges} and returns it.

  \item[static void list\_of\_edges (const \GT{Graph}\& \Param{g},
  const list<edge>\& \Param{edges},
  string\& \Param{result});]
  \strut\\
  Constructs a Tcl list of the edges in \texttt{edges} and
  returns it in \emph{result}.

\end{Cdefinition}


%
% Wrappers for Tcl functions
%

\subsection{Wrappers for common Tcl functions}

The following functions convert C strings to integer, double or
boolean values:

\begin{Cdefinition}

  \item[static int get\_int (GT\_Tcl\_info\& \Param{info}, const
  char* \Param{s}, int\& \Param{result})]
  \strut\\
  Wrapper for the Tcl function \texttt{Tcl\_GetInt}.
  \texttt{GT\_Tcl::get\_int} converts \emph{s} into
  \emph{result}. It returns the return value of
  \texttt{Tcl\_GetInt}.

  \item[static int get\_double (GT\_Tcl\_info\& \Param{info}, const
  char* \Param{s}, double\& \Param{result})]
  \strut\\
  Wrapper for the Tcl function \texttt{Tcl\_GetDouble}.
  \texttt{GT\_Tcl::get\_double} converts \emph{s} into
  \emph{result}. It returns the return value of
  \texttt{Tcl\_GetDouble}.

  \item[static int get\_boolean (GT\_Tcl\_info\& \Param{info}, const
  char* \Param{s}, bool\& \Param{result})]
  \strut\\
  Wrapper for the Tcl function \texttt{Tcl\_GetBoolean}.
  \texttt{GT\_Tcl::get\_boolean} converts \emph{s} into
  \emph{result}. It returns the return value of
  \texttt{Tcl\_GetBoolean}.

  \item[static int get\_boolean (GT\_Tcl\_info\& \Param{info}, const
  char* \Param{s}, int\& \Param{result})]
  \strut\\
  Wrapper for the Tcl function \texttt{Tcl\_GetBoolean}.
  \texttt{GT\_Tcl::get\_boolean} converts \emph{s} into
  \emph{result}. It returns the return value of
  \texttt{Tcl\_GetBoolean}.

\end{Cdefinition}

Example \ref{e:get_double} shows how to use these functions. It
is important to check the returned value; if it is \textbf{not}
\texttt{TCL\_OK}, then \emph{result} does not contain a valid
value.

\begin{example}{e:get_double}%
{Convert a string into an \texttt{double} value using \GT{Tcl::get\_double}}
\begin{verbatim}
int parse (GT_Tcl_info& info, int& index, GT_Tcl_Graph* g)
{
    double result;

    int code = GT_Tcl::get_double (info, info[index], result);
    if (code != TCL_OK) {
        return code;
    }

    // Not do something with result
}
\end{verbatim}
\end{example}

The following wrappers provide an easy way to use the Tcl
function \emph{Tcl\_SplitList} from C++.

\begin{Cdefinition}
  
  \item[static int split\_list (Tcl\_Interp* \Param{interp},
  const char* \Param{name}, list<string>\& \Param{splitted})]
  \strut\\
  
  \item[static int split\_list (Tcl\_Interp* \Param{interp}, const
  char* \Param{name}, int\& \Param{argc}, char**\& \Param{argv})]
  \strut\\
    
\end{Cdefinition}


%
%
%


\section{The class \GTTcl{info}}
\label{s:Tcl_info}

\CSourceCode{}{tcl}{Tcl\_Info}

The standard parameter set for a tcl command are client data, Tcl
interpreter, number of arguments and array aof arguments as in:

\begin{alltt}
int sample_cmd (ClientData client_data,
    Tcl_Interp* interp,
    int argc,
    char* argv[])
\end{alltt}

GraphScript uses a different approach. First, it uses the
\verb|client_data| parameters for its own purposes. Technically,
\verb|client_data| is a pointer to some information which is
stored with the Tcl command. In a C++ interface, this is
typically used to store a pointer to the C++ object which is
associated with the command.

Second, it provides only a single parameter of type
\GTTcl{info} which contains all the above parameters (except
\texttt{client\_data}) plus a bunch of helper functions.

\begin{Cdeclaration}{Tclinfo}{class \GTTcl{info}}
\begin{verbatim}
class GT_Tcl_info
{
    GT_BASE_CLASS (GT_Tcl_info);
        
    GT_VARIABLE (Tcl_Interp*, interp);
    GT_VARIABLE (int, argc);
    GT_VARIABLE (char**, argv);
    GT_COMPLEX_VARIABLE (string, msg);
        
public:

    GT_Tcl_info();
    GT_Tcl_info (Tcl_Interp* interp, int argc, char** argv);
    virtual ~GT_Tcl_info();

    inline const char* operator[] (const int i) const;
    inline char* operator[] (const int i);
    const GT_Key operator() (const int i) const;
    GT_Key operator() (const int i);
    
    bool is_last_arg (const int index) const;
    bool exists (const int index) const;
    bool args_left (const int index, const int n, bool exact = true) const;
    int args_left (const int index) const;
    bool args_left_at_least (const int index, const int n)  const;
    bool args_left_exactly (const int index, const int n) const;
        
    string& msg(); // non-const access to msg
    void msg (const int error);
    void msg (const int error, const int i);
    void msg (const int error, const string& s);
}
\end{verbatim}
\end{Cdeclaration}

Figure \ref{Cdeclaration:Tcl_info} shows the class \texttt{Tcl\_info}.

\begin{Cdefinition}
  
  \item[const char* operator[] (const int \Param{i}) const;]
  \strut\\
  Access argument \emph{i} as a \texttt{char*}. There is also a
  non-const version available. As an example,
\begin{alltt}
void sample (GT_Tcl_info& info)
\{
    for (i=0; i < info.argc(), i++) \{
        const char* a = info[i];
        if (GT::streq (a, "-option") \{
            // Now do something with a
        \}
    \}
\}
\end{alltt}
  
  \item[const char* operator() (const int \Param{i}) const;]
  \strut\\
  Access argument \emph{i} as a \GT{Key} object. There is also a non-const
  version available.

  \begin{notes}
    \item This will create a new object in the global keymapper
    table (see Section \ref{s:Keymapper}).
    \item This access method should \emph{not} be used for
    arbitrary text objects, because this would fill the keymapper
    with lost of one-time-used entries. One reasonable use are
    options like \texttt{-something}.
  \end{notes}

  \item[bool is\_last\_arg (const int \Param{index})]
  \strut\\
  Checks wether the argument \emph{index} is the last argument.
  
  \item[bool exists (const int \Param{index})]
  \strut\\
  Checks wether there exists an argument at \emph{index}.
  
  \item[bool args\_left (const int \Param{index}, const int
  \Param{n}, bool \Param{exact})]
  \strut\\
  Checks wether there are at least (\emph{exact} ==
  \texttt{false}) or exactly (\emph{exact} == \texttt{true})
  \emph{n} arguments left. The following three methods provide a
  more comfortable access to this information.
  
  \item[int args\_left (const int \Param{index})]
  \strut\\
  Returns how many arguments are left after \emph{index}.
  
  \item[bool args\_left\_at\_least (const int \Param{index},
  const int \Param{n}) const]
  \strut\\
  Checks wether there are at least \emph{n} arguments left after
  \emph{index}.
  
  \item[bool args\_left\_exactly (const int \Param{index},
  const int \Param{n}) const]
  \strut\\
  Checks wether there are at least \emph{n} arguments left after
  \emph{index}.

  \item[msg()]
  \strut\\
  Set the return message of the command, as in the following:
\begin{verbatim}
info.msg() = "This is a pity excuse for a better error message.";
\end{verbatim}
  
  \item[void msg (const int \Param{error})]
  \strut\\
  Output Graphlet error message \emph{error}. See also Section
  \ref{s:Error}.
  
  \item[void msg (const int \Param{error}, const int
  \Param{i})]
  \strut\\
  Output Graphlet error message \emph{error} with parameter
  \emph{i}. See also Section \ref{s:Error}.
  
  \item[void msg (const int \Param{error}, const string\&
  \Param{i})]
  \strut\\
  Output Graphlet error message \emph{error} with parameter
  \emph{s}. See also Section \ref{s:Error}.

\end{Cdefinition}

%
% The Tcl Interface of a Graph Algorithm
%

\section{The class \GTTcl{Algorithm}}
\label{s:Tcl_Algorithm}


\GTTcl{Algorithm\_Command} derives from
\GTTcl{Command} and implements a command which works on
graphs.  Consequently, the first parameter of the command is
always a graph.  \GTTcl{Algorithm$<$$>$} is a template
class which constructs a Tcl command from an algorithm.  The Tcl
interface for an algorithm is always derived from (an instance
of) this class.


%
% The Tcl Interface of a Graph Algorithm
%

\subsection{The Tcl Interface of a Graph Algorithm} 
\label{s:GT_Tcl_Interface}

A Tcl interface for a \Graphlet{} algorithm is constructed with
the template class \GTTcl{Algorithm}, as shown in Example
\ref{e:C++:SampleTclAlgorithmClass}. The class
\GTTcl{Algorithm$<$\Param{a}$>$} constructs a generic Tcl
interface for a class \emph{a} which is derived from
\GT{Algorithm}.

\begin{example}%
{e:C++:SampleTclAlgorithmClass}%
{A Sample Tcl Algorithm Interface Class Declaration}
\begin{verbatim}
class Tcl_Sample : public GT_Tcl_Algorithm<Sample>
{
    public:    
    Tcl_Sample (const string& name);
    virtual ~Tcl_Sample ();
    virtual int parse (GT_Tcl_info& info, int& index, GT_Tcl_Graph* g);
};
\end{verbatim}
\end{example}

Note that the so constructed class \texttt{Tcl\_Sample} derives
both from \texttt{Sample} and \GTTcl{Algorithm\_Command}.


\subsection{Required Methods}

The following methods are required for a class \texttt{Tcl\_Sample}
which is derived from \GTTcl{Algorithm$<>$}.

\begin{Cdefinition}
  
  \item[Tcl\_Sample (const string\& name)] This is the
  constructor of the class \texttt{Tcl\_Sample}.  Its parameter
  is the name of the Tcl command, which is usually also the name
  of the algorithm.
  
  \GTTcl{Algorithm$<$$>$}'s constructor takes a single argument
  which is the name of the Tcl Command, so the constructor should
  be declared as

\begin{alltt}
Tcl_Sample::Tcl_Sample (const string& \Param{name}) :
    GT_Tcl_Algorithm<Sample> (\Param{name})
\end{alltt}

  \item[virtual ~Tcl\_Sample()] This is the destructor of the
  class \texttt{Tcl\_Sample}. It must be declared virtual since
  \GTTcl{Algorithm$<>$} has virtual functions.
  
\end{Cdefinition}


\subsection{Optional Methods}

\begin{Cdefinition}
  \item[virtual int parse (GT\_Tcl\_info\& \Param{info},
  int\& \Param{index},
  GT\_Tcl\_Graph* \Param{g})]
  \texttt{parse} is called to parse the argument at \texttt{index}.
\end{Cdefinition}

Also, the methods \texttt{run} and \texttt{check}
from the algorithm class may be overwritten. This is necessary
to convert the results to Tcl format.

\TBD{}


%
% Returning Results
%

\subsection{Returning a result}

The class \GTTcl{Algorithm} provides a member variable
\texttt{result} of type \texttt{string} which holds the Tcl
result of the algorithm.

%
% Returning an error code
%

\subsection{Returning an error code}

To transform the result of an algorithm's \texttt{check} or
\texttt{run} methods into Tcl, proceed as follows:

\begin{enumerate}

  \item Implement \texttt{check} and/or \texttt{run} methods in the 
  class derived from \GTTcl{Algorithm$<$Algorithm$>$}.

  \item These methods execute \texttt{Algorithm::check} respectively.  
  \texttt{Algorithm::check}.

  \item Then convert the output of \texttt{check} or \texttt{run} into 
  a Tcl string and put that into result.

\end{enumerate}




%
% Error Handling
%

\subsection{Error Handling}

The default behavior for the Tcl interface is to return a
Tcl error, which will cause a runtime error unless
\texttt{catch}ed if the check fails. The following piece of Tcl
can be used to prevent a runtime message:

\begin{verbatim}
if [catch { my_algorithm $GT($top,graph) } error_message] {
    tk_dialog .my_errormsg "Error Message" \
        $error_message error 0 "Ok"
}
\end{verbatim}%$
  
\noindent An even better way to solve the above problem is to implement a
dedicated GraphScript command for the \emph{check} phase. This
will also allow for a better user interface for error handling,
e.g. to select the ``bad'' nodes and edges.



%
% Installing \GraphScript{} Commands
%

\section{Installing \GraphScript{} Commands}
Installation and initialization of a \GraphScript{} command takes three steps:
\begin{enumerate}
  \item Create the C++ object.
  \item Install it into the Tcl interpreter.
  \item \texttt{Important:}
  Check the Tcl return code for errors.
\end{enumerate}
\GTTcl{Command}'s 
\texttt{install} method is used to install a command into a Tcl 
interpreter:
\begin{verbatim}
// (1) Create the C++ object
GT_Tcl_Algorithm_Command* sample =
new GT_Tcl_Sample ("sample");
// (2) Install the object in the interpreter
code = sample-$>$install (interp);
// (3) IMPORTANT: check return code.
if (code == TCL_ERROR) {
    return code;
    }
\end{verbatim}

This installation should be placed in the startup file of your
\GraphScript{} interpreter or in the
initialization of your module.



%%%%%%%%%%%%%%%%%%%%%%%%%%%%%%%%%%%%%%%%%%%%%%%%%%%%
%
% Makefiles
%
%%%%%%%%%%%%%%%%%%%%%%%%%%%%%%%%%%%%%%%%%%%%%%%%%%%%

\chapter{Makefiles}
\label{c:Makefiles}

This section describes how \Graphlet{}'s configuration system can be 
used to add makefiles for modules.  The \emph{source code 
distribution} must be installed to take advantage of these features.

\begin{skills}
  This section requires basic knowledge of how \emph{make} programs 
  work.
\end{skills}


%
% Overview of the configuration system
%

\section{Overview of the configuration system}

\Graphlet{} uses \texttt{GNU make} for its configuration.  This 
version of make has some features which are not normally present in 
make programs, such as extended macro processing and 
\emph{if\ldots{}then\ldots{}else} structures.  We chose GNU make over 
preprocessor based systems because the latter one would be harder to 
debug.

One side effect of using GNU make is that we are using the name 
\texttt{GNUmakefile} instead of \texttt{Makefile} or 
\texttt{makefile}.  This has the advantage that any other 
\texttt{make} program will not even try to source the file and report 
syntax errors, but will report that it could not find a \emph{make
file} and exit.  We feel this is a cleaner approach and 
gives better error messages.  Also, our coding standards require to 
use the name \texttt{GNUmakefile} for makefiles.

All of \Graphlet{}'s configuration files are located in 
\texttt{lib/graphlet/config}.  The default installation procedure 
copies the configuration system to
\emph{\Graphlet{} Installation Directory}\texttt{/lib/graphlet/config}.

The configuration system does not only determine the proper compiler 
flags and procedures for installing \Graphlet{} on a particular 
system, but also provides a large set of predefined make variables and 
targets.

To get access to the predefined variables and procedures, every 
\texttt{GNUmakefile} \emph{must} include the file 
\texttt{lib/graphlet/config/common}.  Since the information in 
\texttt{common} relies on information about the site installation, it 
can only be used with a source code distribution.


%
% Anatomy of a \texttt{GNUmakefile}
%

\section{Anatomy of a \texttt{GNUmakefile}}

An example for a \texttt{GNUmakefile} can be found in the algorithms 
module.  A \texttt{GNUmakefile} must at least contain the following:

\begin{enumerate}

  \item  Define of the variable \emph{MODULE}:
  \begin{quote}
    \texttt{MODULE = }\emph{insert the name of the module here}
  \end{quote}
  \emph{MODULE} is the name of the module in which the
  \texttt{GNUmakefile} resides.  

  \item  Define the variable \emph{GRAPHLET\_BASE\_DIR}:
  \begin{quote}
    \texttt{GRAPHLET\_BASE\_DIR=}\emph{insert the relative path to the
    toplevel of \Graphlet{}}
  \end{quote}
  \emph{GRAPHLET\_BASE\_DIR} must be the \texttt{relative} path to the 
  toplevel directory of \Graphlet{}.  For example, if your module is in 
  \texttt{src/}\emph{mymodule}, then set
  \begin{quote}
    \texttt{GRAPHLET\_BASE\_DIR=../..}
  \end{quote}
  
  \item 
  Define a standard target \texttt{all} which is executed when
  no arguments are given to make (which is usually the case):
  \begin{quote}
    \texttt{all:    lib\$(MODULE).a \$(SUBDIRS)}
  \end{quote}
  This defines that the default actions are building the
  object libraries, and recursively traversing the subdirectories.
  
  For a module which has no default actions and no
  subdirectories, use
  \begin{alltt}
.PHONY: all
all:    
        \emph{Insert the actions for target \texttt{all} here}
\end{alltt}
   \noindent The directive \texttt{.PHONY:} states that the following target(s)
    have no dependencies.
  
  \item Include the common configuration definitions:
  \begin{quote}
    \texttt{include\$(\Graphlet{}\_BASE\_DIR)/lib/\Graphlet{}/config/common}
  \end{quote}
  
\end{enumerate}


%
% Standard Variables
%

\section{Standard Variables}

\begin{ttdescription}

  \item[SUBDIRS]  
  The variable \texttt{SUBDIRS} holds a list of subdirectories.
  Most targets will recursively descend into subdirectories.  

  \item[TCLFILES]
  \texttt{TCLFILES} is a list of Tcl files. This target
  \texttt{index} constructs a \texttt{tclIndex}
  file (for autoloading)

  \item[MYCFILES]
  \texttt{MYCFILES} is the list of C/C++ files which
  are not generated by a program (e.g. yacc or lex).
  
  \item[CFILES]
  \texttt{CFILES} is the list of all C/C++ files,
  including those which generated by a program (e.g. yacc or lex).
  
  \item[HFILES]
  \texttt{HFILES} is the list of C/C++ Header files.
  HFILES can be generated from MYCFILES as follows:
  \begin{quote}
    \texttt{HFILES = \$(CFILES:\%.cpp=\%.h)}
  \end{quote}
\end{ttdescription}


\begin{notes}
  \item You must assign values to both \texttt{MYCFILES} and 
  \texttt{CFILES}.
  \item The convention that there is a header file for each C++ file is 
  essential for configuration system to work correctly.
\end{notes}

%
% Standard Targets
%

\section{Standard Targets}

\begin{description}

  \item[\textbf{it}, \texttt{all}]
  \texttt{it} and \texttt{all} must the default targets 
  which are executed when no targets are given to make.
  Both targets \texttt{it} and \texttt{all} are required.

  \item[install.local]
  This target is used to perform installation operations which are
  specific for the local directory. \TBD{}.

\end{description}



%%%%%%%%%%%%%%%%%%%%%%%%%%%%%%%%%%%%%%%%%%
%
% Modules
%
%%%%%%%%%%%%%%%%%%%%%%%%%%%%%%%%%%%%%%%%%%

\chapter{Modules}
\label{c:Modules}

This section describes how to design \Graphlet{} modules.  A
\texttt{module} is a set of C++ and \GraphScript{} files which
implement one or several algorithms. The core of Graphlet itself is 
divided into three modules:

\begin{itemize}
        \item  The module \texttt{gt\_base}, which implements basic and Tcl 
        independend data structures.

        \item  The module \texttt{gt\_tcl}, which implements Graphlet's Tcl 
        interface. Most of \GraphScript{} is implemented here.

        \item  The module \texttt{gt\_algorithms} implements various algorithms.
        
\end{itemize}

Each module has a name, which must be unique throughout the Graphlet 
system.  There are no restrictions on the name, but it is usually a 
good idea to choose a descriptive name.  Later, the header files of a 
module \emph{name} will be installed in a directory 
\texttt{name}\footnote{As a subdirectory of the Graphlet installation 
directory, e.g.\ \texttt{/usr/local/include} on UNIX systems.}, and 
its library will be installed under the name 
\texttt{lib}\emph{name}\texttt{.a}\footnote{On UNIX systems}.  
Graphlet uses the prefix \texttt{gt\_} for its names to prevent name 
clashes with other software packages.

To avoid cluttering the system with too many libraries, modules should 
contain several algorithms, and should possibly further structured 
into submodules.

%
% Guidelines for adding C++ modules
%

\section{Guidelines for adding C++ modules}

\subsection{Example}


An example module can be found in the directory 
\texttt{src/gt\_algorithms} in the Graphlet distribution.

\begin{itemize}

  \item The initialization file of the algorithms module:
  \begin{quote}
    \texttt{src/gt\_algorithms/algorithms.cpp}
  \end{quote}

  \item Header file of the initialization procedure of the algorithms module:
  \begin{quote}
    \texttt{src/gt\_algorithms/algorithms.h}
  \end{quote}

\end{itemize}



\subsection{The Initialization File}

A module \emph{mod} must provide its initialization code in a file
named \texttt{\Param{mod}.cpp} (respectively.\  \texttt{\Param{mod}.h}).
The following headers are required for module initialization files:

\begin{itemize}
  \item \verb|#include <gt_base/Graphlet.h>|
  \item \verb|#include <gt_tcl/Tcl_Algorithm.h>|
  \item \verb|#include <gt_tcl/GraphScript.h>|
\end{itemize}

%
% The Initialization Procedure
%

\subsection{The Initialization Procedure}

The initialization procedure for a module \emph{mod} must be declared 
as follows:

\begin{alltt}
int GT_\Param{mod}_init (Tcl_Interp* interp, GT_GraphScript* graphscript)
\end{alltt}

where \texttt{interp} is the current Tcl interpreter, and 
\texttt{graphscript} is a pointer to the current \GraphScript{} class.  
The initialization procedure must return \texttt{TCL\_OK} if the 
initialization was successful, and \texttt{TCL\_ERROR} otherwise.

\begin{notes}

  \item The naming convention is necessary to use the standard 
  initialization macros with \texttt{application\_init}.

  \item Initialization procedures must \emph{never} make any 
  assumptions on the sequence in which initialization procedures are 
  called.  The only valid assumption is that \Graphlet{}'s 
  initializations are called before any algorithm modules are 
  initialized.

\end{notes}




%
% Guidelines for adding \GraphScript{} modules
%

\section{Guidelines for adding \GraphScript{} modules}

\subsection{Placement}

During \emph{development}, \GraphScript{} files are best kept in the 
developers directory.  This is best done with autoloading (see the Tcl 
documentation for details).  For \emph{inclusion in the source code 
distribution}, \GraphScript{} files should be placed in the directory 
\texttt{lib/graphscript} within \Graphlet{}, or in subdirectories of 
\texttt{lib/graphscript}.


\subsection{Filenames in \texttt{lib/graphscript}}

The filenames should reflect the name of the module:
\begin{itemize}

  \item If there is only one \GraphScript{} file, it should have the 
  same name as the module (with suffix \texttt{.tcl}).
  
  \item If there are several files, organize them in a subdirectory 
  with the name of the module.
  
\end{itemize}


\subsection{GNUmakefile modifications}

To add files to the \GraphScript{} library, edit the file

\begin{alltt} 
  lib/graphscript/GNUmakefile
\end{alltt}

\noindent and add the files to the list of files in the variable 
\texttt{TCLFILES}.  It is also neccessary to re-create the Tcl index 
after new files were added or changed.  This can be done with the 
following command:

\begin{quote}
  \texttt{gmake index}
\end{quote}

\noindent where \texttt{gmake} is GNU make.

\begin{notes}
  \item
  Subdirectories do not need to have their own GNUmakefile's.  
  Instead, add all files to \texttt{lib/graphscript/GNUmakefile}.  
  Remember to use relative paths.
\end{notes}


\subsection{Initialization}

There are two ways to initialize a \GraphScript{} module:

\begin{enumerate}
  \item 
  Tcl will execute any statements at the top level of a
  Tcl file when it loads the file. Put initialization code
  here.
  
  \item 
  \emph{(Preferred)}
  Implement an initialization procedure in Tcl (preferably called 
  \GT{init\_\Param{name}} for module \emph{name}) and execute that 
  procedure in the initialization procedure of the C++ code.  
  Here is an example:
  
\begin{verbatim}
code = Tcl_Eval (interp, "GT_name_init");
if (code == TCL_ERROR) {
    return code;
}
\end{verbatim}

  The Tcl initialization procedure should come at the end of the
  C++ module initialization, after all \GraphScript{} commands have
  been installed.

\end{enumerate}


%
% How to determine what is installed
%

\section{How to determine what is installed}

Before features are used which are implemented in an optional module, 
you need to check whether the module is present.  Generally, 
\Graphlet{} installations have the freedom to install or de-install 
any module.  Therefore, it is \textbf{illegal} to make any a priory 
assumptions about installed modules.  It is however possible to obtain 
information on whether a specific feature is installed:

\begin{description}

  \item[C++]
  For each module \emph{name}, \Graphlet{} defines a preprocessor 
  symbol \GT{MODULE\_\Param{NAME}} (note the capital letters).  This 
  can be used to write code which is only executed when a specific 
  module exists:
  
\begin{alltt}
#ifdef GT_MODULE_NAME
  \ldots \emph{put code here that needs module \Param{name} here} \ldots
#endif
\end{alltt}


\item[\GraphScript{}]

  Tcl goes even one step further than C++ and allows runtime testing 
  for installed commands with \texttt{info commands}.  Here is an 
  example how to use that:

\begin{alltt}
if \{ [info commands cmd_name] != \{\} \} \{
  \ldots \emph{Put code here that needs command cmd_name} \ldots
\}
\end{alltt}

  \begin{note}
        It is generally a good idea to add menu entries only for those 
        commands which are actually installed.
  \end{note}

\end{description}



%%%%%%%%%%%%%%%%%%%%%%%%%%%%%%%%%%%%%%%%%%
%
% Building \GraphScript{} Interpreters
%
%%%%%%%%%%%%%%%%%%%%%%%%%%%%%%%%%%%%%%%%%%

\section{Building \GraphScript{} Interpreters}

For a quick take, copy and modify \Graphlet{}'s \texttt{graphscript.cpp}.

After new \GraphScript{} commands have been implemented, it is
necessary to build a new interpreter which contains these new
commands. Technically, this the same procedure as needed for a
standard Tcl interpreter, with \GraphScript{} as an additional
module. 

\begin{note}
  Tcl/Tk now supports dynamic loading.  We will change the 
  initialization procedure at some point in the future to change 
  support dynamic loading.  Stay tuned.  There should be no major 
  problems unless you use global variables, in which case there will.
\end{note}

%
%
%

\section{The \texttt{main} procedure}

Generally, a main procedure for a \GraphScript{} interpreter should 
have the following form:

\begin{verbatim}
int main (int argc, char **argv) 
{
    return GT_GraphScript::gt_main (argc, argv, application_init);    
}
\end{verbatim}

\GT{GraphScript:gt\_main} is standard method which calls Tk's
\texttt{Tk\_Main} procedure (see the Tk documentation for
details) and parses the command line in \texttt{argc} and
\texttt{argv} for GraphScript specific options.
\texttt{application\_init} must be provided by the application
and initializes \GraphScript{} and algorithm modules.

\begin{notes}
  \item It is legal to use the procedure \texttt{Tk\_main}
  instead of \GT{GraphScript:gt\_main}.
\end{notes}
%
%
%

\section{The procedure \texttt{application\_init}}

Each interpreter must provide a procedure \texttt{application\_init} 
which must have the following form:

\begin{verbatim}
//
// application_init (Tcl_interp* interp)
//
// interp is the current Tcl interpreter.
//

static int application_init (Tcl_Interp *interp)
{
    //
    // Create a GraphScript handler.
    // For customization, replace the class GT_GraphScript
    // by a derived class.
    //
    
    GT_GraphScript* graphscript = new GT_GraphScript (interp);
    
    //
    // Initialize Tcl, Tk and GraphScript.
    //
    // code holds the Tcl return code.
    //
    
    int code = graphscript->application_init (interp);

    //
    // Check for an error. This step is mandatory.
    //

    if (code == TCL_ERROR) {
        return code;
    }

    //
    // Initialize other algorithm modules
    //

#ifdef GT_MODULE_ALGORITHMS
    GT_ADD_MODULE(gt_algorithms)
#endif

#ifdef GT_MODULE_LSD
    GT_ADD_MODULE(gt_lsd)
#endif GT_MODULE_LSD
    
#ifdef GT_MODULE_GRID_ALGORITHMS
    GT_ADD_MODULE(gt_grid_algorithms)
#endif GT_MODULE_LSD
    
    return code;
}
\end{verbatim}


\begin{notes}

  \item We recommend the above syntax for the initialization of other 
  modules, provided that these modules conform to the standards 
  described in the module documentation.
  
  \item \Graphlet{}'s configuration system automatically defines a 
  macro \GT{MODULE\_\Param{NAME}} for each installed module \emph{name} 
  (note the capital letters).
  
  \item The macro \GT{ADD\_MODULE} is provided by \Graphlet{} and 
  performs all actions necessary to initialize a standard module.  It 
  will exit the procedure and \texttt{return} \texttt{TCL\_ERROR} if 
  the module initialization fails.
  
  \item \texttt{application\_init} must return a Tcl error code, that 
  is \texttt{TCL\_OK} for a successful return or \texttt{TCL\_ERROR} 
  otherwise.
  
  \item Because Tcl is a C library, \texttt{application\_init} must be 
  a procedure or a static member function.
  
\end{notes}


%
%
%

\section{Linking the program}

\begin{WhoCanSkipThis}  
  This section is meant as a guideline for developers who do not use
  the binary distribution. Developers who use the source code
  distribution may use \Graphlet{}'s configuration system to create
  makefiles.  
\end{WhoCanSkipThis}

Link our code with the following libraries:

\begin{enumerate}

  \item[\Graphlet{} libraries]
  \texttt{\emph{MODULES} -lgt\_tcl -lgt\_base} where \emph{MODULES} is 
  the list of modules as used in \texttt{application\_init} above.
  
  \item[Tcl/Tk libraries]
  \texttt{-ltk -ltcl}

  \item[LEDA libraries (release 3.4)]  
  \texttt{-lP -lG -lL}
  
  \item[X11 and system libraries]
  This depends on your operating system, flavor of X11 and
  local installation policy. Here are some recommendations:
  \begin{itemize}
    \item On solaris systems (2.4/2.5), we recommend
    \texttt{-L/usr/openwin/lib -lX11 -lsocket -lnsl -ldl -lm}.
    
    \item On SunOS systems (4.1.*), we recommend
    \texttt{-L/usr/openwin/lib -lX11 -lm} 
    
    \item On Linux systems, we recommend
    \texttt{-L/usr/X11/lib -lX11 -ldl -lm}
  \end{itemize}
  
\end{enumerate}

\begin{notes}
  \item If you are linking with LEDA 3.3.*, you need to add the library 
  \texttt{-lWx}.
  \item The order in which libraries are linked \texttt{is} important.  
  We recommend to link libraries in the order indicated above.
\end{notes}


\end{document}

%%% Local Variables: 
%%% mode: latex
%%% End: 
