\documentclass[notitlepage,twoside,fleqn]{article}
\usepackage[article]{Graphlet}

\begin{document}

%%%%%%%%%%%%%%%%%%%%%%%%%%%%%%%%%%%%%%%%%%
%
% Extension Modules
%
%%%%%%%%%%%%%%%%%%%%%%%%%%%%%%%%%%%%%%%%%%

% This file is generated automatically
\newcommand{\GraphletVersion}{1}


\title{
  Graphlet Packages Specification
  \\*[3.0cm]
  {\textbf{DRAFT VERSION}}
  }

\author{Michael Himsolt}

\maketitle

\begin{abstract}
  This document defines the specifications for Graphlet packages.
  A Graphlet package is a software module, for example a graph
  drawing algorithm, which is a separate unit within the Graphlet
  system.
\end{abstract}

\tableofcontents


\section{General}

The package must be conform to the graphlet coding standards, as
described in the Graphlet Coding Standards Manual.


\section{GraphScript}

Each algorithm must have a GraphScript interface defined. This
interface must implement at least the following:

\begin{itemize}
  \item A GraphScript command for the alogrithm.
  \item A \texttt{check} method which makes sure that the
  algorithm is only used with an appropriate graph. See the
  GraphScript manual for details.
  \item A parser for the parameter(s) of the algorithm.
\end{itemize}

All non-trivial algorithms should have their top level
implemented in GraphScript. This has several advantages. First,
it is easy to replace parts of the algorithm, even without
recompiling the algorithm. Second, since the user interface is
implemented with GraphScript, the integration of the algorithm
and the the user interface becomes easier. Especially, each
submodule may have its own user interface. Third, it is easier to
modify the general outline of the algorithm, and write batch
scripts which test these configurations on benchmark graphs.


\section{User Interface}

Each algorithm should at least provide a pop-up window for the
parameter settings. Optional user interface elements include

\begin{itemize}
  
  \item Additional windows, e.g.\ tool palettes.
  
  As an example, the user might select one or more edges and then
  click at a button in the palette to assign a constraint to this
  edge. This can of course also be implemented as menus, but
  menus are less convenient since the user must pull down the
  menu first.
  
  \item Display the result of the algorithm. This can be done by
  color coding some edges, output a message in the footer or
  opening a dialog window. This is especially important for error
  messages.

  \item Support for algorithm animation.

\end{itemize}


\section{Test Suite}

\subsection{Test Graphs}

Each algorithm should be accompanied by a set of test graphs:

\begin{itemize}
  
  \item Graphs with errors in it. These graphs \emph{must
    trigger} an error message and \emph{must not} crash the
  system.
  
  \item Graphs without errors. These should be used to check that
  the algorithm runs correctly. It is important to realize that
  these graphs should not display the best face of the algorithm,
  but shoudl test \emph{all} features of the algorithm.
  
  \item Sample graphs which show the beauty of the algorithm.
  
\end{itemize}
  
\subsection{Test Programs}

Each algorithm should be accompanied by test programs, which
check that the algorithm runs correct. Ideally, the test programs
should run the test graphs.
  
The test programs should be written in GraphScript or optionally
C++.  GraphScript is preferred since testing and exception
handling is easier. For example, the scripts may use Tcl's
\texttt{catch} command to track errors.


\section{Documentation}

\emph{Programmer documentation} should be done in TeX with Graphlet's
LaTeX2e style as supplied in the directory \texttt{doc} in
the Graphlet source code directory.

\emph{End user documentation} can either be supplied as online
documentation (e.g.\ as HTML files), or as printed manuals.


\end{document}

%%% Local Variables: 
%%% mode: latex
%%% End: 
